\section{Related Literature}\label{sec-lit}

\subsection{Background on Counterfactual
Explanations}\label{background-on-counterfactual-explanations}

\cite{wachter2017counterfactual, joshi2019realistic, altmeyer2024faithful}

\subsection{Learning Representations}\label{learning-representations}

\begin{quote}
For example, joint-energy models
\end{quote}

\subsection{Generalization and
Robustness}\label{generalization-and-robustness}

\cite{sauer2021counterfactual} generate counterfactual images for MNIST
and ImageNet through independent mechanisms (IM): each IM learns
class-conditional input distributions over a specific lower-dimensional,
semantically meaningful factor, such as \emph{texture}, \emph{shape} and
\emph{background}. The demonstrate that using these generated
counterfactuals during classifier training improves model robustness.
Similarly, \cite{abbasnejad2020counterfactual} argue that
counterfactuals represent potentially useful training data in machine
learning, especially in supervised settings where inputs may be
reasonably mapped to multiple outputs. They, too, demonstrate the
augmenting the training data of image classifiers can improve
generalization.

\cite{teney2020learning} propose an approach using counterfactuals in
training that does not rely on data augmentation: they argue that
counterfactual pairs typically already exist in training datasets.
Specifically, their approach relies on, firstly, identifying similar
input samples with different annotations and, secondly, ensuring that
the gradient of the classifier aligns with the vector between pairs of
counterfactual inputs using the cosine distance as a loss function
(referred to as \emph{gradient supervision}) (\textbf{\emph{this might
be useful for our task as well}}). In the natural language processing
(NLP) domain, counterfactuals have similarly been used to improve models
through data augmentation: \cite{wu2021polyjuice}, propose POLYJUICE, a
general-purpose counterfactual generator for language models. They
demonstrate empirically that augmenting training data through POLYJUICE
counterfactuals improves robustness in a number of NLP tasks.

\subsection{Link to Adversarial
Training}\label{link-to-adversarial-training}

\cite{freiesleben2022intriguing} propose two definitional differences
between Adversarial Examples (AE) and Counterfactual Explanations (CE):
firstly, and more importantly according to the authors, the term AE
implies missclassification, which is not the case for CE
(\textbf{\emph{this might be a useful notion for use to distinguish
between adversarials and explanations during training}}); secondly, they
argue that closeness plays a more critical role in the context of CE but
confess that even counterfactuals that are not close might be relevant
explanations. \cite{pawelczyk2022exploring} show that CE and AE are
equivalent under certain conditions and derive upper bounds on the
distances between them.

\subsection{Closely Related}\label{closely-related}

\cite{guo2023counternet} are the first to propose end-to-end training
pipeline that includes counterfactual explanations as part of the
training prodeduce. In particular, they propose a specific network
architecture that includes a predictor and CE generator network
(\textbf{\emph{akin a GAN?}}), where the parameters of the CE generator
network are learnable. Counterfactuals are generated during each
training iteration and fed back to the predictor network
(\textbf{\emph{here we are aligned}}). In contrast, we impose no
restrictions on the neural network architecture at all.
(\textbf{\emph{to ensure the one-hot encoding of categorical features is
maintained, they simple use softmax (might be interesting for CE.jl)}})
Interestingly, the authors find that their approach is sensitive to the
choice of the loss function: only MSE seems to lead to good performance.
They also demonstrate theoretically, that the objective function is
difficult to optimize due to divergent gradients and suffers from poor
adversarial robustness. (\textbf{\emph{because partial gradients with
respect to the classification loss component and the counterfactual
validity component point in opposite directions}}). To mitigate these
issues, the authors use block-wise gradient descent: they first update
with respect to classification loss and then use a second update with
respect to the other loss components (\textbf{\emph{this might be useful
for our task as well}}). \cite{ross2021learning} propose a way to train
models that are guaranteed to provide recourse for individuals with high
probability. The approach builds on adversarial training
(\textbf{\emph{here we are aligned}}), where in this context adversarial
examples are actively encouraged to exist, but only target attacks with
respect to the positive class. The proposed method allows for imposing a
set of actionable recourse ex-ante: for example, users can impose
mutability constraints for features (\textbf{\emph{here we are
aligned}}). (\textbf{\emph{To solve their objective function more
efficiently, they use a first-order Taylor approximation to approximate
the recourse loss component (might be applicable in our case)}})

\cite{luu2023counterfactual} introduce Counterfactual Adversarial
Training (CAT) with intention of improving generalization and robustness
of language models. Specifically, they propose to proceed as follows:
firstly, identify training samples that are subject to high predictive
uncertainty (entropy); secondly, generate counterfactual explanations
for those samples; and, finally, finetune the model on the augmented
dataset that includes the generated counterfactuals.
