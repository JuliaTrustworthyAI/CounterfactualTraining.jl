% Options for packages loaded elsewhere
% Options for packages loaded elsewhere
\PassOptionsToPackage{unicode}{hyperref}
\PassOptionsToPackage{hyphens}{url}
\PassOptionsToPackage{dvipsnames,svgnames,x11names}{xcolor}
%
\documentclass[
]{article}
\usepackage{xcolor}
\usepackage{amsmath,amssymb}
\setcounter{secnumdepth}{5}
\usepackage{iftex}
\ifPDFTeX
  \usepackage[T1]{fontenc}
  \usepackage[utf8]{inputenc}
  \usepackage{textcomp} % provide euro and other symbols
\else % if luatex or xetex
  \usepackage{unicode-math} % this also loads fontspec
  \defaultfontfeatures{Scale=MatchLowercase}
  \defaultfontfeatures[\rmfamily]{Ligatures=TeX,Scale=1}
\fi
\usepackage{lmodern}
\ifPDFTeX\else
  % xetex/luatex font selection
  \setmainfont[]{Latin Modern Roman}
  \setmathfont[]{Latin Modern Math}
\fi
% Use upquote if available, for straight quotes in verbatim environments
\IfFileExists{upquote.sty}{\usepackage{upquote}}{}
\IfFileExists{microtype.sty}{% use microtype if available
  \usepackage[]{microtype}
  \UseMicrotypeSet[protrusion]{basicmath} % disable protrusion for tt fonts
}{}
\makeatletter
\@ifundefined{KOMAClassName}{% if non-KOMA class
  \IfFileExists{parskip.sty}{%
    \usepackage{parskip}
  }{% else
    \setlength{\parindent}{0pt}
    \setlength{\parskip}{6pt plus 2pt minus 1pt}}
}{% if KOMA class
  \KOMAoptions{parskip=half}}
\makeatother
% Make \paragraph and \subparagraph free-standing
\makeatletter
\ifx\paragraph\undefined\else
  \let\oldparagraph\paragraph
  \renewcommand{\paragraph}{
    \@ifstar
      \xxxParagraphStar
      \xxxParagraphNoStar
  }
  \newcommand{\xxxParagraphStar}[1]{\oldparagraph*{#1}\mbox{}}
  \newcommand{\xxxParagraphNoStar}[1]{\oldparagraph{#1}\mbox{}}
\fi
\ifx\subparagraph\undefined\else
  \let\oldsubparagraph\subparagraph
  \renewcommand{\subparagraph}{
    \@ifstar
      \xxxSubParagraphStar
      \xxxSubParagraphNoStar
  }
  \newcommand{\xxxSubParagraphStar}[1]{\oldsubparagraph*{#1}\mbox{}}
  \newcommand{\xxxSubParagraphNoStar}[1]{\oldsubparagraph{#1}\mbox{}}
\fi
\makeatother


\usepackage{longtable,booktabs,array}
\usepackage{calc} % for calculating minipage widths
% Correct order of tables after \paragraph or \subparagraph
\usepackage{etoolbox}
\makeatletter
\patchcmd\longtable{\par}{\if@noskipsec\mbox{}\fi\par}{}{}
\makeatother
% Allow footnotes in longtable head/foot
\IfFileExists{footnotehyper.sty}{\usepackage{footnotehyper}}{\usepackage{footnote}}
\makesavenoteenv{longtable}
\usepackage{graphicx}
\makeatletter
\newsavebox\pandoc@box
\newcommand*\pandocbounded[1]{% scales image to fit in text height/width
  \sbox\pandoc@box{#1}%
  \Gscale@div\@tempa{\textheight}{\dimexpr\ht\pandoc@box+\dp\pandoc@box\relax}%
  \Gscale@div\@tempb{\linewidth}{\wd\pandoc@box}%
  \ifdim\@tempb\p@<\@tempa\p@\let\@tempa\@tempb\fi% select the smaller of both
  \ifdim\@tempa\p@<\p@\scalebox{\@tempa}{\usebox\pandoc@box}%
  \else\usebox{\pandoc@box}%
  \fi%
}
% Set default figure placement to htbp
\def\fps@figure{htbp}
\makeatother


% definitions for citeproc citations
\NewDocumentCommand\citeproctext{}{}
\NewDocumentCommand\citeproc{mm}{%
  \begingroup\def\citeproctext{#2}\cite{#1}\endgroup}
\makeatletter
 % allow citations to break across lines
 \let\@cite@ofmt\@firstofone
 % avoid brackets around text for \cite:
 \def\@biblabel#1{}
 \def\@cite#1#2{{#1\if@tempswa , #2\fi}}
\makeatother
\newlength{\cslhangindent}
\setlength{\cslhangindent}{1.5em}
\newlength{\csllabelwidth}
\setlength{\csllabelwidth}{3em}
\newenvironment{CSLReferences}[2] % #1 hanging-indent, #2 entry-spacing
 {\begin{list}{}{%
  \setlength{\itemindent}{0pt}
  \setlength{\leftmargin}{0pt}
  \setlength{\parsep}{0pt}
  % turn on hanging indent if param 1 is 1
  \ifodd #1
   \setlength{\leftmargin}{\cslhangindent}
   \setlength{\itemindent}{-1\cslhangindent}
  \fi
  % set entry spacing
  \setlength{\itemsep}{#2\baselineskip}}}
 {\end{list}}
\usepackage{calc}
\newcommand{\CSLBlock}[1]{\hfill\break\parbox[t]{\linewidth}{\strut\ignorespaces#1\strut}}
\newcommand{\CSLLeftMargin}[1]{\parbox[t]{\csllabelwidth}{\strut#1\strut}}
\newcommand{\CSLRightInline}[1]{\parbox[t]{\linewidth - \csllabelwidth}{\strut#1\strut}}
\newcommand{\CSLIndent}[1]{\hspace{\cslhangindent}#1}



\setlength{\emergencystretch}{3em} % prevent overfull lines

\providecommand{\tightlist}{%
  \setlength{\itemsep}{0pt}\setlength{\parskip}{0pt}}



 


\usepackage[title]{appendix}
\usepackage{placeins}
\usepackage{amsthm}
\theoremstyle{plain}
\newtheorem{proposition}{Proposition}[section]
\theoremstyle{definition}
\newtheorem{definition}{Definition}[section]
\theoremstyle{definition}
\newtheorem{example}{Example}[section]
\theoremstyle{plain}
\usepackage{arxiv}
\usepackage{orcidlink}
\usepackage{amsmath}
\usepackage[T1]{fontenc}
\usepackage{placeins}
\usepackage{siunitx}
\makeatletter
\@ifpackageloaded{tcolorbox}{}{\usepackage[skins,breakable]{tcolorbox}}
\@ifpackageloaded{fontawesome5}{}{\usepackage{fontawesome5}}
\definecolor{quarto-callout-color}{HTML}{909090}
\definecolor{quarto-callout-note-color}{HTML}{0758E5}
\definecolor{quarto-callout-important-color}{HTML}{CC1914}
\definecolor{quarto-callout-warning-color}{HTML}{EB9113}
\definecolor{quarto-callout-tip-color}{HTML}{00A047}
\definecolor{quarto-callout-caution-color}{HTML}{FC5300}
\definecolor{quarto-callout-color-frame}{HTML}{acacac}
\definecolor{quarto-callout-note-color-frame}{HTML}{4582ec}
\definecolor{quarto-callout-important-color-frame}{HTML}{d9534f}
\definecolor{quarto-callout-warning-color-frame}{HTML}{f0ad4e}
\definecolor{quarto-callout-tip-color-frame}{HTML}{02b875}
\definecolor{quarto-callout-caution-color-frame}{HTML}{fd7e14}
\makeatother
\makeatletter
\@ifpackageloaded{caption}{}{\usepackage{caption}}
\AtBeginDocument{%
\ifdefined\contentsname
  \renewcommand*\contentsname{Table of contents}
\else
  \newcommand\contentsname{Table of contents}
\fi
\ifdefined\listfigurename
  \renewcommand*\listfigurename{List of Figures}
\else
  \newcommand\listfigurename{List of Figures}
\fi
\ifdefined\listtablename
  \renewcommand*\listtablename{List of Tables}
\else
  \newcommand\listtablename{List of Tables}
\fi
\ifdefined\figurename
  \renewcommand*\figurename{Figure}
\else
  \newcommand\figurename{Figure}
\fi
\ifdefined\tablename
  \renewcommand*\tablename{Table}
\else
  \newcommand\tablename{Table}
\fi
}
\@ifpackageloaded{float}{}{\usepackage{float}}
\floatstyle{ruled}
\@ifundefined{c@chapter}{\newfloat{codelisting}{h}{lop}}{\newfloat{codelisting}{h}{lop}[chapter]}
\floatname{codelisting}{Listing}
\newcommand*\listoflistings{\listof{codelisting}{List of Listings}}
\usepackage{amsthm}
\theoremstyle{remark}
\AtBeginDocument{\renewcommand*{\proofname}{Proof}}
\newtheorem*{remark}{Remark}
\newtheorem*{solution}{Solution}
\newtheorem{refremark}{Remark}[section]
\newtheorem{refsolution}{Solution}[section]
\makeatother
\makeatletter
\makeatother
\makeatletter
\@ifpackageloaded{caption}{}{\usepackage{caption}}
\@ifpackageloaded{subcaption}{}{\usepackage{subcaption}}
\makeatother
\newcounter{quartocalloutnteno}
\newcommand{\quartocalloutnte}[1]{\refstepcounter{quartocalloutnteno}\label{#1}}
\usepackage{bookmark}
\IfFileExists{xurl.sty}{\usepackage{xurl}}{} % add URL line breaks if available
\urlstyle{same}
\hypersetup{
  pdftitle={Supplementary Appendix},
  pdfauthor={Patrick Altmeyer; Aleksander Buszydlik; Arie van Deursen; Cynthia C. S. Liem},
  pdfkeywords={Counterfactual Training, Counterfactual
Explanations, Algorithmic Recourse, Explainable AI, Representation
Learning},
  colorlinks=true,
  linkcolor={blue},
  filecolor={Maroon},
  citecolor={Blue},
  urlcolor={Blue},
  pdfcreator={LaTeX via pandoc}}


\newcommand{\runninghead}{A Preprint }
\renewcommand{\runninghead}{Counterfactual Training }
\title{Supplementary Appendix}
\def\asep{\\\\\\ } % default: all authors on same column
\def\asep{\And }
\author{\textbf{Patrick
Altmeyer}~\orcidlink{0000-0003-4726-8613}\\Faculty of Electrical
Engineering, Mathematics and Computer Science\\Delft University of
Technology\\\\\href{mailto:p.altmeyer@tudelft.nl}{p.altmeyer@tudelft.nl}\asep\textbf{Aleksander
Buszydlik}\\Faculty of Electrical Engineering, Mathematics and Computer
Science\\Delft University of Technology\\\\\asep\textbf{Arie van
Deursen}\\Faculty of Electrical Engineering, Mathematics and Computer
Science\\Delft University of Technology\\\\\asep\textbf{Cynthia C. S.
Liem}\\Faculty of Electrical Engineering, Mathematics and Computer
Science\\Delft University of Technology\\\\}
\date{}
\begin{document}
\maketitle
{\bfseries \emph Keywords}
\def\sep{\textbullet\ }
Counterfactual Training \sep Counterfactual
Explanations \sep Algorithmic Recourse \sep Explainable AI \sep 
Representation Learning



\begin{appendices}

\section{Notation}\label{notation}

Below we provide an overview of some notation used frequently throughout
the paper:

\begin{itemize}
\tightlist
\item
  \(y^+\): The target class and also the index of the target class.
\item
  \(y^-\): The non-target class and also the index of non-the target
  class.
\item
  \(\mathbf{x}\): a single training sample.
\item
  \(\mathbf{x}^\prime\): a counterfactual.
\item
  \(\mathbf{x}^+\): a training sample in the target class
  (ground-truth).
\item
  \(\mathbf{y}^+\): The one-hot encoded output vector for the target
  class.
\item
  \(\theta\): Model parameters (unspecified).
\item
  \(\Theta\): Matrix of parameters.
\item
  \(\mathbf{M}(\cdot)\): linear predictions (logits) of the classifier.
\end{itemize}

\subsection{Other Technical Details}\label{other-technical-details}

Maximum mean discrepancy is defined as follows,

\begin{equation}\phantomsection\label{eq-mmd}{
\begin{aligned}
\text{MMD}({X}^\prime,\tilde{X}^\prime) &= \frac{1}{m(m-1)}\sum_{i=1}^m\sum_{j\neq i}^m k(x_i,x_j) \\ &+ \frac{1}{n(n-1)}\sum_{i=1}^n\sum_{j\neq i}^n k(\tilde{x}_i,\tilde{x}_j) \\ &- \frac{2}{mn}\sum_{i=1}^m\sum_{j=1}^n k(x_i,\tilde{x}_j)
\end{aligned}
}\end{equation}

where \(k(\cdot,\cdot)\) is a kernel function
(\citeproc{ref-gretton2012kernel}{Gretton et al. 2012}). We make use of
a Gaussian kernel with a constant length-scale parameter of \(0.5\). In
our implementation, Equation~\ref{eq-mmd} is by default applied to the
entire subset of the training data for which \(y=y^+\).

\section{Technical Details of Our
Approach}\label{technical-details-of-our-approach}

\subsection{Generating Counterfactuals through Gradient
Descent}\label{sec-app-ce}

In this section, we provide some background on gradient-based
counterfactual generators (Section~\ref{sec-app-ce-background}) and
discuss how we define convergence in this context
(Section~\ref{sec-app-conv}).

\subsubsection{Background}\label{sec-app-ce-background}

Gradient-based counterfactual search was originally proposed by Wachter,
Mittelstadt, and Russell
(\citeproc{ref-wachter2017counterfactual}{2017}). It generally solves
the following unconstrained objective,

\[
\begin{aligned}
\min_{\mathbf{z}^\prime \in \mathcal{Z}^L} \left\{  {\text{yloss}(\mathbf{M}_{\theta}(g(\mathbf{z}^\prime)),\mathbf{y}^+)}+ \lambda {\text{cost}(g(\mathbf{z}^\prime)) }  \right\} 
\end{aligned} 
\]

where \(g: \mathcal{Z} \mapsto \mathcal{X}\) is an invertible function
that maps from the \(L\)-dimensional counterfactual state space to the
feature space and \(\text{cost}(\cdot)\) denotes one or more penalties
that are used to induce certain properties of the counterfactual
outcome. As above, \(\mathbf{y}^+\) denotes the target output and
\(\mathbf{M}_{\theta}(\mathbf{x})\) returns the logit predictions of the
underlying classifier for \(\mathbf{x}=g(\mathbf{z})\).

For all generators used in this work we use standard logit crossentropy
loss for \(\text{yloss}(\cdot)\). All generators also penalize the
distance (\(\ell_1\)-norm) of counterfactuals from their original
factual state. For \emph{Generic} and \emph{ECCo}, we have
\(\mathcal{Z}:=\mathcal{X}\) and
\(g(\mathbf{z})=g(\mathbf{z})^{-1}=\mathbf{z}\), that is counterfactual
are searched directly in the feature space. Conversely, \emph{REVISE}
traverses the latent space of a variational autoencoder (VAE) fitted to
the training data, where \(g(\cdot)\) corresponds to the decoder
(\citeproc{ref-joshi2019realistic}{Joshi et al. 2019}). In addition to
the distance penalty, \emph{ECCo} uses an additional penalty component
that regularizes the energy associated with the counterfactual,
\(\mathbf{x}^\prime\) (\citeproc{ref-altmeyer2024faithful}{Altmeyer et
al. 2024}).

\subsubsection{Convergence}\label{sec-app-conv}

An important consideration when generating counterfactual explanations
using gradient-based methods is how to define convergence. Two common
choices are to 1) perform gradient descent over a fixed number of
iterations \(T\), or 2) conclude the search as soon as the predicted
probability for the target class has reached a pre-determined threshold,
\(\tau\):
\(\mathcal{S}(\mathbf{M}_\theta(\mathbf{x}^\prime))[y^+] \geq \tau\). We
prefer the latter for our purposes, because it explicitly defines
convergence in terms of the black-box model, \(\mathbf{M}(\mathbf{x})\).

Defining convergence in this way allows for a more intuitive
interpretation of the resulting counterfactual outcomes than with fixed
\(T\). Specifically, it allows us to think of counterfactuals as
explaining `high-confidence' predictions by the model for the target
class \(y^+\). Depending on the context and application, different
choices of \(\tau\) can be considered as representing `high-confidence'
predictions.

\subsection{Protecting Mutability Constraints with Linear
Classifiers}\label{sec-app-constraints}

In \textbf{?@sec-constraints} we explain that to avoid penalizing
implausibility that arises due to mutability constraints, we impose a
point mass prior on \(p(\mathbf{x})\) for the corresponding feature. We
argue in \textbf{?@sec-constraints} that this approach induces models to
be less sensitive to immutable features and demonstrate this empirically
in \textbf{?@sec-experiments}. Below we derive the analytical results in
Prp.\textasciitilde{}\ref{prp-mtblty}.

\begin{proof}
Let \(d_{\text{mtbl}}\) and \(d_{\text{immtbl}}\) denote some mutable
and immutable feature, respectively. Suppose that
\(\mu_{y^-,d_{\text{immtbl}}} < \mu_{y^+,d_{\text{immtbl}}}\) and
\(\mu_{y^-,d_{\text{mtbl}}} > \mu_{y^+,d_{\text{mtbl}}}\), where
\(\mu_{k,d}\) denotes the conditional sample mean of feature \(d\) in
class \(k\). In words, we assume that the immutable feature tends to
take lower values for samples in the non-target class \(y^-\) than in
the target class \(y^+\). We assume the opposite to hold for the mutable
feature.

Assuming multivariate Gaussian class densities with common diagonal
covariance matrix \(\Sigma_k=\Sigma\) for all \(k \in \mathcal{K}\), we
have for the log likelihood ratio between any two classes
\(k,m \in \mathcal{K}\) (\citeproc{ref-hastie2009elements}{Hastie,
Tibshirani, and Friedman 2009}):

\begin{equation}\phantomsection\label{eq-loglike}{
\log \frac{p(k|\mathbf{x})}{p(m|\mathbf{x})}=\mathbf{x}^\intercal \Sigma^{-1}(\mu_{k}-\mu_{m})  + \text{const}
}\end{equation}

By independence of \(x_1,...,x_D\), the full log-likelihood ratio
decomposes into:

\begin{equation}\phantomsection\label{eq-loglike-decomp}{
\log \frac{p(k|\mathbf{x})}{p(m|\mathbf{x})} = \sum_{d=1}^D \frac{\mu_{k,d}-\mu_{m,d}}{\sigma_{d}^2} x_{d} + \text{const}
}\end{equation}

By the properties of our classifier (\emph{multinomial logistic
regression}), we have:

\begin{equation}\phantomsection\label{eq-multi}{
\log \frac{p(k|\mathbf{x})}{p(m|\mathbf{x})} = \sum_{d=1}^D \left( \theta_{k,d} - \theta_{m,d} \right)x_d + \text{const}
}\end{equation}

where \(\theta_{k,d}=\Theta[k,d]\) denotes the coefficient on feature
\(d\) for class \(k\).

Based on Equation~\ref{eq-loglike-decomp} and Equation~\ref{eq-multi} we
can identify that
\((\mu_{k,d}-\mu_{m,d}) \propto (\theta_{k,d} - \theta_{m,d})\) under
the assumptions we made above. Hence, we have that
\((\theta_{y^-,d_{\text{immtbl}}} - \theta_{y^+,d_{\text{immtbl}}}) < 0\)
and
\((\theta_{y^-,d_{\text{mtbl}}} - \theta_{y^+,d_{\text{mtbl}}}) > 0\)

Let \(\mathbf{x}^\prime\) denote some randomly chosen individual from
class \(y^-\) and let \(y^+ \sim p(y)\) denote the randomly chosen
target class. Then the partial derivative of the contrastive divergence
penalty \textbf{?@eq-div} with respect to coefficient \(\theta_{y^+,d}\)
is equal to

\begin{equation}\phantomsection\label{eq-grad}{
\frac{\partial}{\partial\theta_{y^+,d}} \left(\text{div}(\mathbf{x}^+,\mathbf{x^\prime},\mathbf{y};\theta)\right) = \frac{\partial}{\partial\theta_{y^+,d}} \left( \left(-\mathbf{M}_\theta(\mathbf{x}^+)[y^+]\right) - \left(-\mathbf{M}_\theta(\mathbf{x}^\prime)[y^+]\right) \right) = x_{d}^\prime - x^+_{d}
}\end{equation}

and equal to zero everywhere else.

Since \((\mu_{y^-,d_{\text{immtbl}}} < \mu_{y^+,d_{\text{immtbl}}})\) we
are more likely to have
\((x_{d_{\text{immtbl}}}^\prime - x^+_{d_{\text{immtbl}}}) < 0\) than
vice versa at initialization. Similarly, we are more likely to have
\((x_{d_{\text{mtbl}}}^\prime - x^+_{d_{\text{mtbl}}}) > 0\) since
\((\mu_{y^-,d_{\text{mtbl}}} > \mu_{y^+,d_{\text{mtbl}}})\).

This implies that if we do not protect feature \(d_{\text{immtbl}}\),
the contrastive divergence penalty will decrease
\(\theta_{y^-,d_{\text{immtbl}}}\) thereby exacerbating the existing
effect
\((\theta_{y^-,d_{\text{immtbl}}} - \theta_{y^+,d_{\text{immtbl}}}) < 0\).
In words, not protecting the immutable feature would have the
undesirable effect of making the classifier more sensitive to this
feature, in that it would be more likely to predict class \(y^-\) as
opposed to \(y^+\) for lower values of \(d_{\text{immtbl}}\).

By the same rationale, the contrastive divergence penalty can generally
be expected to increase \(\theta_{y^-,d_{\text{mtbl}}}\) exacerbating
\((\theta_{y^-,d_{\text{mtbl}}} - \theta_{y^+,d_{\text{mtbl}}}) > 0\).
In words, this has the effect of making the classifier more sensitive to
the mutable feature, in that it would be more likely to predict class
\(y^-\) as opposed to \(y^+\) for higher values of \(d_{\text{mtbl}}\).

Thus, our proposed approach of protecting feature \(d_{\text{immtbl}}\)
has the net affect of decreasing the classifier's sensitivity to the
immutable feature relative to the mutable feature (i.e.~no change in
sensitivity for \(d_{\text{immtbl}}\) relative to increased sensitivity
for \(d_{\text{mtbl}}\)).
\end{proof}

\subsection{Domain Constraints}\label{domain-constraints}

We apply domain constraints on counterfactuals during training and
evaluation. There are at least two good reasons for doing so. Firstly,
within the context of explainability and algorithmic recourse,
real-world attributes are often domain constrained: the \emph{age}
feature, for example, is lower bounded by zero and upper bounded by the
maximum human lifespan. Secondly, domain constraints help mitigate
training instabilities commonly associated with energy-based modelling
(\citeproc{ref-grathwohl2020your}{Grathwohl et al. 2020};
\citeproc{ref-altmeyer2024faithful}{Altmeyer et al. 2024}).

For our image datasets, features are pixel values and hence the domain
is constrained by the lower and upper bound of values that pixels can
take depending on how they are scaled (in our case \([-1,1]\)). For all
other features \(d\) in our synthetic and tabular datasets, we
automatically infer domain constraints
\([x_d^{\text{LB}},x_d^{\text{UB}}]\) as follows,

\begin{equation}\phantomsection\label{eq-domain}{
\begin{aligned}
x_d^{\text{LB}} &= \arg\min_{x_d} \{\mu_d - n_{\sigma_d}\sigma_d, \arg \min_{x_d} x_d\} \\
x_d^{\text{UB}} &= \arg\max_{x_d} \{\mu_d + n_{\sigma_d}\sigma_d, \arg \max_{x_d} x_d\} 
\end{aligned}
}\end{equation}

where \(\mu_d\) and \(\sigma_d\) denote the sample mean and standard
deviation of feature \(d\). We set \(n_{\sigma_d}=3\) across the board
but higher values and hence wider bounds may be appropriate depending on
the application.

\subsection{Training Hyperparameters}\label{sec-app-training}

Note~\ref{nte-train-default} presents the default hyperparameters used
during training.

\begin{tcolorbox}[enhanced jigsaw, toptitle=1mm, colframe=quarto-callout-note-color-frame, toprule=.15mm, colbacktitle=quarto-callout-note-color!10!white, bottomrule=.15mm, coltitle=black, arc=.35mm, opacitybacktitle=0.6, bottomtitle=1mm, title={Note \ref*{nte-train-default}: Training Phase}, opacityback=0, titlerule=0mm, rightrule=.15mm, leftrule=.75mm, breakable, left=2mm, colback=white]

\quartocalloutnte{nte-train-default} 

\begin{itemize}
\tightlist
\item
  Meta Parameters:

  \begin{itemize}
  \tightlist
  \item
    Generator: \texttt{ecco}
  \item
    Model: \texttt{mlp}
  \end{itemize}
\item
  Model:

  \begin{itemize}
  \tightlist
  \item
    Activation: \texttt{relu}
  \item
    No.~Hidden: \texttt{32}
  \item
    No.~Layers: \texttt{1}
  \end{itemize}
\item
  Training Parameters:

  \begin{itemize}
  \tightlist
  \item
    Burnin: \texttt{0.0}
  \item
    Class Loss: \texttt{logitcrossentropy}
  \item
    Convergence: \texttt{threshold}
  \item
    Generator Parameters:

    \begin{itemize}
    \tightlist
    \item
      Decision Threshold: \texttt{0.75}
    \item
      \(\lambda_{\text{cst}}\): \texttt{0.001}
    \item
      \(\lambda_{\text{egy}}\): \texttt{5.0}
    \item
      Learning Rate: \texttt{0.25}
    \item
      Maximum Iterations: \texttt{30}
    \item
      Optimizer: \texttt{sgd}
    \item
      Type: \texttt{ECCo}
    \end{itemize}
  \item
    \(\lambda_{\text{adv}}\): \texttt{0.25}
  \item
    \(\lambda_{\text{clf}}\): \texttt{1.0}
  \item
    \(\lambda_{\text{div}}\): \texttt{0.5}
  \item
    \(\lambda_{\text{reg}}\): \texttt{0.1}
  \item
    Learning Rate: \texttt{0.001}
  \item
    No.~Counterfactuals: \texttt{1000}
  \item
    No.~Epochs: \texttt{100}
  \item
    Objective: \texttt{full}
  \item
    Optimizer: \texttt{adam}
  \end{itemize}
\end{itemize}

\end{tcolorbox}

\subsection{Evaluation Details}\label{sec-app-eval}

For all of our evaluations, we proceed as follows: for each experiment
setting we generate multiple counterfactuals (``No.~Counterfactuals''),
randomly choosing the factual and target class each time
(Note~\ref{nte-eval-default}). We do this across multiple rounds
(``No.~Runs'') with different random seeds to account for stochasticity
(Note~\ref{nte-eval-default}). This is in line with standard practice in
the related literature on CE. Note~\ref{nte-eval-default} presents the
default hyperparameters used during evaluation. For our final results
presented in the main paper, we rely on held out test sets to sample
factuals (and outputs for our performance metrics). For tuning purposes
we rely on training or validation sets.

\subsubsection{Robust Accuracy}\label{robust-accuracy}

To evaluate robust accuracy (Acc.\(^*\)), we use the Fast Gradient Sign
Method (FGSM) to perturb test samples
(\citeproc{ref-goodfellow2014explaining}{Goodfellow, Shlens, and Szegedy
2015}). For the main results, we have set the perturbation size to
\(\epsilon=0.03\). We have also tested other perturbation sizes, as well
as randomly perturbed data. Although not reported here, we have
consistently found strong outperformance of CT compared to the weak
baseline.

\begin{tcolorbox}[enhanced jigsaw, toptitle=1mm, colframe=quarto-callout-note-color-frame, toprule=.15mm, colbacktitle=quarto-callout-note-color!10!white, bottomrule=.15mm, coltitle=black, arc=.35mm, opacitybacktitle=0.6, bottomtitle=1mm, title={Note \ref*{nte-eval-default}: Evaluation Phase}, opacityback=0, titlerule=0mm, rightrule=.15mm, leftrule=.75mm, breakable, left=2mm, colback=white]

\quartocalloutnte{nte-eval-default} 

\begin{itemize}
\tightlist
\item
  Counterfactual Parameters:

  \begin{itemize}
  \tightlist
  \item
    Convergence: \texttt{threshold}
  \item
    Decision Threshold: \texttt{0.95}
  \item
    Generator Parameters:

    \begin{itemize}
    \tightlist
    \item
      Decision Threshold: \texttt{0.75}
    \item
      \(\lambda_{\text{cst}}\): \texttt{0.001}
    \item
      \(\lambda_{\text{egy}}\): \texttt{5.0}
    \item
      Learning Rate: \texttt{0.25}
    \item
      Maximum Iterations: \texttt{30}
    \item
      Optimizer: \texttt{sgd}
    \item
      Type: \texttt{ECCo}
    \end{itemize}
  \item
    Maximum Iterations: \texttt{50}
  \item
    No.~Individuals: \texttt{100}
  \item
    No.~Runs: \texttt{5}
  \end{itemize}
\end{itemize}

\end{tcolorbox}

\FloatBarrier

\section{Details on Main Experiments}\label{sec-app-main}

\subsection{Final Hyperparameters}\label{final-hyperparameters}

As discussed \textbf{?@sec-experiments}, CT is sensitive to certain
hyperparameter choices. We study the effect of many hyperparameters
extensively in Section~\ref{sec-app-grid}. For the main results, we tune
a small set of key hyperparameters (Section~\ref{sec-app-tune}). The
final choices for the main results are presented for each data set in
Table~\ref{tbl-final-params} along with training, test and batch sizes.

\begin{table}

\caption{\label{tbl-final-params}Final hyperparameters used for the main
results presented in \textbf{?@sec-experiments}. Any hyperparameter not
shown here is set to its default value (Note~\ref{nte-train-default}).}

\centering{

\begin{tabular}{cccccccc}
  \toprule
  \textbf{Data} & \textbf{No. Train} & \textbf{No. Test} & \textbf{Batchsize} & \textbf{Domain} & \textbf{Decision Threshold} & \textbf{No. Counterfactuals} & \textbf{$\lambda_{\text{reg}}$} \\\midrule
  LS & 3600 & 600 & 30 & none & 0.5 & 1000 & 0.01 \\
  Circ & 3600 & 600 & 30 & none & 0.5 & 1000 & 0.5 \\
  Moon & 3600 & 600 & 30 & none & 0.9 & 1000 & 0.25 \\
  OL & 3600 & 600 & 30 & none & 0.5 & 1000 & 0.25 \\\midrule
  Adult & 26049 & 5010 & 1000 & none & 0.75 & 5000 & 0.25 \\
  CH & 16504 & 3101 & 1000 & none & 0.5 & 5000 & 0.25 \\
  Cred & 10617 & 1923 & 1000 & none & 0.5 & 5000 & 0.25 \\
  GMSC & 13371 & 2474 & 1000 & none & 0.5 & 5000 & 0.5 \\
  MNIST & 11000 & 2000 & 1000 & (-1.0, 1.0) & 0.5 & 5000 & 0.01 \\
\end{tabular}

}

\end{table}%

\subsection{Final Results}\label{sec-app-final-results}

Plus/minus two standard deviations of bootstrap estimates.

\subsubsection{Robust Performance Plots}\label{robust-performance-plots}

\subsection{Confidence Intervals}\label{confidence-intervals}

\begin{table}

\caption{\label{tbl-ci}Confidence intervals}

\centering{

\begin{tabular}{lccccc}
  \toprule
  \textbf{Variable} & \textbf{Data} & \textbf{Full} & \textbf{Vanilla} & \textbf{LB} & \textbf{UB} \\\midrule
  Cost $(-\%)$ & LS & 3.82 & 4.44 & -0.7 & -0.56 \\
  $ \text{IP}^* $ $(-\%)$ & LS & 0.1 & 0.23 & -0.14 & -0.12 \\
  $ \text{IP} $ $(-\%)$ & LS & 2.41 & 3.4 & -1.04 & -0.94 \\
  Cost $(-\%)$ & Circ & 0.67 & 1.23 & -0.58 & -0.53 \\\midrule
  $ \text{IP}^* $ $(-\%)$ & Circ & 0.0 & 0.0 & -0.01 & -0.0 \\
  $ \text{IP} $ $(-\%)$ & Circ & 1.03 & 2.36 & -1.37 & -1.29 \\
  Cost $(-\%)$ & Moon & 1.55 & 1.6 & -0.08 & -0.01 \\
  $ \text{IP}^* $ $(-\%)$ & Moon & 0.02 & 0.02 & -0.01 & -0.0 \\
  $ \text{IP} $ $(-\%)$ & Moon & 1.36 & 1.71 & -0.38 & -0.32 \\
  Cost $(-\%)$ & OL & 1.62 & 2.63 & -1.15 & -0.81 \\\midrule
  $ \text{IP}^* $ $(-\%)$ & OL & 0.12 & 0.09 & -0.01 & 0.05 \\
  $ \text{IP} $ $(-\%)$ & OL & 4.49 & 4.44 & -0.03 & 0.13 \\
  Cost $(-\%)$ & Adult & 2.26 & 2.2 & -0.22 & 0.28 \\
  $ \text{IP}^* $ $(-\%)$ & Adult & 0.07 & 0.11 & -0.06 & -0.01 \\
  $ \text{IP} $ $(-\%)$ & Adult & 15.03 & 15.15 & -0.68 & 0.26 \\
  Cost $(-\%)$ & CH & 1.46 & 2.46 & -1.1 & -0.89 \\
  $ \text{IP}^* $ $(-\%)$ & CH & 0.02 & 0.06 & -0.06 & -0.04 \\
  $ \text{IP} $ $(-\%)$ & CH & 6.61 & 7.52 & -1.17 & -0.63 \\
  Cost $(-\%)$ & Cred & 2.68 & 2.29 & 0.16 & 0.63 \\
  $ \text{IP}^* $ $(-\%)$ & Cred & 0.03 & 0.06 & -0.05 & -0.01 \\
  $ \text{IP} $ $(-\%)$ & Cred & 19.31 & 22.03 & -3.69 & -1.74 \\
  Cost $(-\%)$ & GMSC & 1.14 & 3.05 & -2.45 & -1.77 \\
  $ \text{IP}^* $ $(-\%)$ & GMSC & 0.02 & 0.07 & -0.06 & -0.04 \\
  $ \text{IP} $ $(-\%)$ & GMSC & 6.19 & 8.09 & -2.4 & -1.49 \\
  Cost $(-\%)$ & MNIST & 77.04 & 68.67 & -3.47 & 18.34 \\
  $ \text{IP}^* $ $(-\%)$ & MNIST & 0.04 & 0.04 & -0.1 & 0.09 \\
  $ \text{IP} $ $(-\%)$ & MNIST & 258.83 & 278.54 & -30.49 & -7.64 \\
\end{tabular}

}

\end{table}%

\subsection{Qualitative Findings for Image
Data}\label{qualitative-findings-for-image-data}

Figure~\ref{fig-mnist} shows much more plausible (faithful)
counterfactuals for a model with CT than the model with conventional
training (Figure~\ref{fig-mnist-vanilla}).

\begin{figure}

\begin{minipage}{0.46\linewidth}

\begin{figure}[H]

\centering{

\pandocbounded{\includegraphics[keepaspectratio]{../../paper/figures/mnist_mlp.png}}

}

\caption{\label{fig-mnist}Counterfactual images for \emph{MLP} with
counterfactual training. Factual images are shown on the diagonal, with
the corresponding counterfactual for each target class (columns) in that
same row. The underlying generator, \emph{ECCo}, aims to generate
counterfactuals that are faithful to the model
(\citeproc{ref-altmeyer2024faithful}{Altmeyer et al. 2024}).}

\end{figure}%

\end{minipage}%
%
\begin{minipage}{0.09\linewidth}
~\end{minipage}%
%
\begin{minipage}{0.46\linewidth}

\begin{figure}[H]

\centering{

\pandocbounded{\includegraphics[keepaspectratio]{../../paper/figures/mnist_mlp_vanilla.png}}

}

\caption{\label{fig-mnist-vanilla}The same setup, factuals, model
architecture and generator as in Figure~\ref{fig-mnist}, but the model
was trained conventionally.}

\end{figure}%

\end{minipage}%

\end{figure}%

\FloatBarrier

\section{Grid Searches}\label{sec-app-grid}

To assess the hyperparameter sensitivity of our proposed training regime
we ran multiple large grid searches for all of our synthetic datasets.
We have grouped these grid searches into multiple categories:

\begin{enumerate}
\def\labelenumi{\arabic{enumi}.}
\tightlist
\item
  \textbf{Generator Parameters} (Section~\ref{sec-app-grid-gen}):
  Investigates the effect of changing hyperparameters that affect the
  counterfactual outcomes during the training phase.
\item
  \textbf{Penalty Strengths} (Section~\ref{sec-app-grid-pen}):
  Investigates the effect of changing the penalty strengths in out
  proposed objective (\textbf{?@eq-obj}).
\item
  \textbf{Other Parameters} (Section~\ref{sec-app-grid-train}):
  Investigates the effect of changing other training parameters,
  including the total number of generated counterfactuals in each epoch.
\end{enumerate}

We begin by summarizing the high-level findings in
Section~\ref{sec-app-grid-hl}. For each of the categories,
Section~\ref{sec-app-grid-gen} to Section~\ref{sec-app-grid-train} then
present all details including the exact parameter grids, average
predictive performance outcomes and key evaluation metrics for the
generated counterfactuals.

\subsection{Evaluation Details}\label{evaluation-details}

To measure predictive performance, we compute the accuracy and F1-score
for all models on test data (Table~\ref{tbl-acc-gen},
Table~\ref{tbl-acc-pen}, Table~\ref{tbl-acc-train}). With respect to
explanatory performance, we report here our findings for the
(im)plausibility and cost of counterfactuals at test time. Since the
computation of our proposed divergence metric (\textbf{?@eq-impl-div})
is memory-intensive, we rely on the distance-based metric for the grid
searches. For the counterfactual evaluation, we draw factual samples
from the training data for the grid searches to avoid data leakage with
respect to our final results reported in the body of the paper.
Specifically, we want to avoid choosing our default hyperparameters
based on results on the test data. Since we are optimizing for
explainability, not predictive performance, we still present test
accuracy and F1-scores.

\subsubsection{Predictive Performance}\label{predictive-performance}

We find that CT is associated with little to no decrease in average
predictive performance for our synthetic datasets: test accuracy and
F1-scores decrease by at most \textasciitilde1 percentage point, but
generally much less (Table~\ref{tbl-acc-gen}, Table~\ref{tbl-acc-pen},
Table~\ref{tbl-acc-train}). Variation across hyperparameters is
negligible as indicated by small standard deviations for these metrics
across the board.

\subsubsection{Counterfactual Outcomes}\label{sec-app-grid-hl}

Overall, we find that counterfactual training achieves it key objectives
consistently across all hyperparameter settings and also broadly across
datasets: plausibility is improved by up to 60 percent (\%) for the
\emph{Circles} data (e.g.
Figure~\ref{fig-grid-gen_params-plaus-circles}), 25-30\% for the
\emph{Moons} data (e.g. Figure~\ref{fig-grid-gen_params-plaus-moons})
and 10-20\% for the \emph{Linearly Separable} data (e.g.
Figure~\ref{fig-grid-gen_params-plaus-lin_sep}). At the same time, the
average costs of faithful counterfactuals are reduced in many cases by
around 20-25\% for \emph{Circles} (e.g.
Figure~\ref{fig-grid-gen_params-cost-circles}) and up to 50\% for
\emph{Moons} (e.g. Figure~\ref{fig-grid-gen_params-cost-moons}). For the
\emph{Linearly Separable} data, costs are generally increased although
typically by less than 10\% (e.g.
Figure~\ref{fig-grid-gen_params-cost-lin_sep}), which reflects a common
tradeoff between costs and plausibility
(\citeproc{ref-altmeyer2024faithful}{Altmeyer et al. 2024}).

We do observe strong sensitivity to certain hyperparameters, with clear
an manageable patterns. Concerning generator parameters, we firstly find
that using \emph{REVISE} to generate counterfactuals during training
typically yields the worst outcomes out of all generators, often leading
to a substantial decrease in plausibility. This finding can be
attributed to the fact that \emph{REVISE} effectively assigns the task
of learning plausible explanations from the model itself to a surrogate
VAE. In other words, counterfactuals generated by \emph{REVISE} are less
faithful to the model that \emph{ECCo} and \emph{Generic}, and hence we
would expect them to be a less effective and, in fact, potentially
detrimental role in our training regime. Secondly, we observe that
allowing for a higher number of maximum steps \(T\) for the
counterfactual search generally yields better outcomes. This is
intuitive, because it allows more counterfactuals to reach maturity in
any given iteration. Looking in particular at the results for
\emph{Linearly Separable}, it seems that higher values for \(T\) in
combination with higher decision thresholds (\(\tau\)) yields the best
results when using \emph{ECCo}. But depending on the degree of class
separability of the underlying data, a high decision-threshold can also
affect results adversely, as evident from the results for the
\emph{Overlapping} data (Figure~\ref{fig-grid-gen_params-plaus-over}):
here we find that CT generally fails to achieve its objective because
only a tiny proportion of counterfactuals ever reaches maturity.

Regarding penalty strengths, we find that the strength of the energy
regularization, \(\lambda_{\text{reg}}\) is a key hyperparameter, while
sensitivity with respect to \(\lambda_{\text{div}}\) and
\(\lambda_{\text{adv}}\) is much less evident. In particular, we observe
that not regularizing energy enough or at all typically leads to poor
performance in terms of decreased plausibility and increased costs, in
particular for \emph{Circles} (Figure~\ref{fig-grid-pen-plaus-circles}),
\emph{Linearly Separable} (Figure~\ref{fig-grid-pen-plaus-lin_sep}) and
\emph{Overlapping} (Figure~\ref{fig-grid-pen-plaus-over}). High values
of \(\lambda_{\text{reg}}\) can increase the variability in outcomes, in
particular when combined with high values for \(\lambda_{\text{div}}\)
and \(\lambda_{\text{adv}}\), but this effect is less pronounced.

Finally, concerning other hyperparameters we observe that the
effectiveness and stability of CT is positively associated with the
number of counterfactuals generated during each training epoch, in
particular for \emph{Circles}
(Figure~\ref{fig-grid-train-plaus-circles}) and \emph{Moons}
(Figure~\ref{fig-grid-train-plaus-moons}). We further find that a higher
number of training epochs is beneficial as expected, where we tested
training models for 50 and 100 epochs. Interestingly, we find that it is
not necessary to employ CT during the entire training phase to achieve
the desired improvements in explainability: specifically, we have tested
training models conventionally during the first half of training before
switching to CT after this initial burn-in period.

\subsection{Generator Parameters}\label{sec-app-grid-gen}

The hyperparameter grid with varying generator parameters during
training is shown in Note~\ref{nte-gen-params-final-run-train}. The
corresponding evaluation grid used for these experiments is shown in
Note~\ref{nte-gen-params-final-run-eval}.

\begin{tcolorbox}[enhanced jigsaw, toptitle=1mm, colframe=quarto-callout-note-color-frame, toprule=.15mm, colbacktitle=quarto-callout-note-color!10!white, bottomrule=.15mm, coltitle=black, arc=.35mm, opacitybacktitle=0.6, bottomtitle=1mm, title={Note \ref*{nte-gen-params-final-run-train}: Training Phase}, opacityback=0, titlerule=0mm, rightrule=.15mm, leftrule=.75mm, breakable, left=2mm, colback=white]

\quartocalloutnte{nte-gen-params-final-run-train} 

\begin{itemize}
\tightlist
\item
  Generator Parameters:

  \begin{itemize}
  \tightlist
  \item
    Decision Threshold: \texttt{0.75,\ 0.9,\ 0.95}
  \item
    \(\lambda_{\text{egy}}\): \texttt{0.1,\ 0.5,\ 5.0,\ 10.0,\ 20.0}
  \item
    Maximum Iterations: \texttt{5,\ 25,\ 50}
  \end{itemize}
\item
  Generator: \texttt{ecco,\ generic,\ revise}
\item
  Model: \texttt{mlp}
\item
  Training Parameters:

  \begin{itemize}
  \tightlist
  \item
    Objective: \texttt{full,\ vanilla}
  \end{itemize}
\end{itemize}

\end{tcolorbox}

\begin{tcolorbox}[enhanced jigsaw, toptitle=1mm, colframe=quarto-callout-note-color-frame, toprule=.15mm, colbacktitle=quarto-callout-note-color!10!white, bottomrule=.15mm, coltitle=black, arc=.35mm, opacitybacktitle=0.6, bottomtitle=1mm, title={Note \ref*{nte-gen-params-final-run-eval}: Evaluation Phase}, opacityback=0, titlerule=0mm, rightrule=.15mm, leftrule=.75mm, breakable, left=2mm, colback=white]

\quartocalloutnte{nte-gen-params-final-run-eval} 

\begin{itemize}
\tightlist
\item
  Generator Parameters:

  \begin{itemize}
  \tightlist
  \item
    \(\lambda_{\text{egy}}\): \texttt{0.1,\ 0.5,\ 1.0,\ 5.0,\ 10.0}
  \end{itemize}
\end{itemize}

\end{tcolorbox}

\subsubsection{Predictive Performance}\label{predictive-performance-1}

Predictive performance measures for this grid search are shown in
Table~\ref{tbl-acc-gen}.

\begin{longtable}{ccccc}

\caption{\label{tbl-acc-gen}Predictive performance measures by dataset
and objective averaged across training-phase parameters
(Note~\ref{nte-gen-params-final-run-train}) and evaluation-phase
parameters (Note~\ref{nte-gen-params-final-run-eval}).}

\tabularnewline

  \toprule
  \textbf{Dataset} & \textbf{Variable} & \textbf{Objective} & \textbf{Mean} & \textbf{Se} \\\midrule
  \endfirsthead
  \toprule
  \textbf{Dataset} & \textbf{Variable} & \textbf{Objective} & \textbf{Mean} & \textbf{Se} \\\midrule
  \endhead
  \bottomrule
  \multicolumn{5}{r}{Continuing table below.}\\
  \bottomrule
  \endfoot
  \endlastfoot
  Circ & Accuracy & Full & 1.0 & 0.0 \\
  Circ & Accuracy & Vanilla & 1.0 & 0.0 \\
  Circ & F1-score & Full & 1.0 & 0.0 \\
  Circ & F1-score & Vanilla & 1.0 & 0.0 \\
  LS & Accuracy & Full & 1.0 & 0.0 \\
  LS & Accuracy & Vanilla & 1.0 & 0.0 \\
  LS & F1-score & Full & 1.0 & 0.0 \\
  LS & F1-score & Vanilla & 1.0 & 0.0 \\
  Moon & Accuracy & Full & 1.0 & 0.0 \\
  Moon & Accuracy & Vanilla & 1.0 & 0.0 \\
  Moon & F1-score & Full & 1.0 & 0.0 \\
  Moon & F1-score & Vanilla & 1.0 & 0.0 \\
  OL & Accuracy & Full & 0.91 & 0.0 \\
  OL & Accuracy & Vanilla & 0.92 & 0.0 \\
  OL & F1-score & Full & 0.91 & 0.0 \\
  OL & F1-score & Vanilla & 0.92 & 0.0 \\\bottomrule

\end{longtable}

\subsubsection{Plausibility}\label{plausibility}

The results with respect to the plausibility measure are shown in
Figure~\ref{fig-grid-gen_params-plaus-circles} to
Figure~\ref{fig-grid-gen_params-plaus-over}.

\begin{figure}

\centering{

\includegraphics[width=0.8\linewidth,height=\textheight,keepaspectratio]{../../paper/experiments/output/final_run/gen_params/mlp/circles/evaluation/results/ce/decision_threshold_exper---lambda_energy_exper---maxiter_exper---maxiter---decision_threshold_exper/plausibility_distance_from_target.png}

}

\caption{\label{fig-grid-gen_params-plaus-circles}Average outcomes for
the plausibility measure across hyperparameters. This shows the \%
change from the baseline model for the distance-based implausibility
metric (\textbf{?@eq-impl-dist}). Boxplots indicate the variation across
evaluation runs and test settings (varying parameters for \emph{ECCo}).
Data: Circles.}

\end{figure}%

\begin{figure}

\centering{

\includegraphics[width=0.8\linewidth,height=\textheight,keepaspectratio]{../../paper/experiments/output/final_run/gen_params/mlp/lin_sep/evaluation/results/ce/decision_threshold_exper---lambda_energy_exper---maxiter_exper---maxiter---decision_threshold_exper/plausibility_distance_from_target.png}

}

\caption{\label{fig-grid-gen_params-plaus-lin_sep}Average outcomes for
the plausibility measure across hyperparameters. This shows the \%
change from the baseline model for the distance-based implausibility
metric (\textbf{?@eq-impl-dist}). Boxplots indicate the variation across
evaluation runs and test settings (varying parameters for \emph{ECCo}).
Data: Linearly Separable.}

\end{figure}%

\begin{figure}

\centering{

\includegraphics[width=0.8\linewidth,height=\textheight,keepaspectratio]{../../paper/experiments/output/final_run/gen_params/mlp/moons/evaluation/results/ce/decision_threshold_exper---lambda_energy_exper---maxiter_exper---maxiter---decision_threshold_exper/plausibility_distance_from_target.png}

}

\caption{\label{fig-grid-gen_params-plaus-moons}Average outcomes for the
plausibility measure across hyperparameters. This shows the \% change
from the baseline model for the distance-based implausibility metric
(\textbf{?@eq-impl-dist}). Boxplots indicate the variation across
evaluation runs and test settings (varying parameters for \emph{ECCo}).
Data: Moons.}

\end{figure}%

\begin{figure}

\centering{

\includegraphics[width=0.8\linewidth,height=\textheight,keepaspectratio]{../../paper/experiments/output/final_run/gen_params/mlp/over/evaluation/results/ce/decision_threshold_exper---lambda_energy_exper---maxiter_exper---maxiter---decision_threshold_exper/plausibility_distance_from_target.png}

}

\caption{\label{fig-grid-gen_params-plaus-over}Average outcomes for the
plausibility measure across hyperparameters. This shows the \% change
from the baseline model for the distance-based implausibility metric
(\textbf{?@eq-impl-dist}). Boxplots indicate the variation across
evaluation runs and test settings (varying parameters for \emph{ECCo}).
Data: Overlapping.}

\end{figure}%

\subsubsection{Cost}\label{cost}

The results with respect to the cost measure are shown in
Figure~\ref{fig-grid-gen_params-cost-circles} to
Figure~\ref{fig-grid-gen_params-cost-over}.

\begin{figure}

\centering{

\includegraphics[width=0.8\linewidth,height=\textheight,keepaspectratio]{../../paper/experiments/output/final_run/gen_params/mlp/circles/evaluation/results/ce/decision_threshold_exper---lambda_energy_exper---maxiter_exper---maxiter---decision_threshold_exper/distance.png}

}

\caption{\label{fig-grid-gen_params-cost-circles}Average outcomes for
the cost measure across hyperparameters. This shows the \% change from
the baseline model for the distance-based cost metric
(\citeproc{ref-wachter2017counterfactual}{Wachter, Mittelstadt, and
Russell 2017}). Boxplots indicate the variation across evaluation runs
and test settings (varying parameters for \emph{ECCo}). Data: Circles.}

\end{figure}%

\begin{figure}

\centering{

\includegraphics[width=0.8\linewidth,height=\textheight,keepaspectratio]{../../paper/experiments/output/final_run/gen_params/mlp/lin_sep/evaluation/results/ce/decision_threshold_exper---lambda_energy_exper---maxiter_exper---maxiter---decision_threshold_exper/distance.png}

}

\caption{\label{fig-grid-gen_params-cost-lin_sep}Average outcomes for
the cost measure across hyperparameters. This shows the \% change from
the baseline model for the distance-based cost metric
(\citeproc{ref-wachter2017counterfactual}{Wachter, Mittelstadt, and
Russell 2017}). Boxplots indicate the variation across evaluation runs
and test settings (varying parameters for \emph{ECCo}). Data: Linearly
Separable.}

\end{figure}%

\begin{figure}

\centering{

\includegraphics[width=0.8\linewidth,height=\textheight,keepaspectratio]{../../paper/experiments/output/final_run/gen_params/mlp/moons/evaluation/results/ce/decision_threshold_exper---lambda_energy_exper---maxiter_exper---maxiter---decision_threshold_exper/distance.png}

}

\caption{\label{fig-grid-gen_params-cost-moons}Average outcomes for the
cost measure across hyperparameters. This shows the \% change from the
baseline model for the distance-based cost metric
(\citeproc{ref-wachter2017counterfactual}{Wachter, Mittelstadt, and
Russell 2017}). Boxplots indicate the variation across evaluation runs
and test settings (varying parameters for \emph{ECCo}). Data: Moons.}

\end{figure}%

\begin{figure}

\centering{

\includegraphics[width=0.8\linewidth,height=\textheight,keepaspectratio]{../../paper/experiments/output/final_run/gen_params/mlp/over/evaluation/results/ce/decision_threshold_exper---lambda_energy_exper---maxiter_exper---maxiter---decision_threshold_exper/distance.png}

}

\caption{\label{fig-grid-gen_params-cost-over}Average outcomes for the
cost measure across hyperparameters. This shows the \% change from the
baseline model for the distance-based cost metric
(\citeproc{ref-wachter2017counterfactual}{Wachter, Mittelstadt, and
Russell 2017}). Boxplots indicate the variation across evaluation runs
and test settings (varying parameters for \emph{ECCo}). Data:
Overlapping.}

\end{figure}%

\subsection{Penalty Strengths}\label{sec-app-grid-pen}

The hyperparameter grid with varying penalty strengths during training
is shown in Note~\ref{nte-pen-final-run-train}. The corresponding
evaluation grid used for these experiments is shown in
Note~\ref{nte-pen-final-run-eval}.

\begin{tcolorbox}[enhanced jigsaw, toptitle=1mm, colframe=quarto-callout-note-color-frame, toprule=.15mm, colbacktitle=quarto-callout-note-color!10!white, bottomrule=.15mm, coltitle=black, arc=.35mm, opacitybacktitle=0.6, bottomtitle=1mm, title={Note \ref*{nte-pen-final-run-train}: Training Phase}, opacityback=0, titlerule=0mm, rightrule=.15mm, leftrule=.75mm, breakable, left=2mm, colback=white]

\quartocalloutnte{nte-pen-final-run-train} 

\begin{itemize}
\tightlist
\item
  Generator: \texttt{ecco,\ generic,\ revise}
\item
  Model: \texttt{mlp}
\item
  Training Parameters:

  \begin{itemize}
  \tightlist
  \item
    \(\lambda_{\text{adv}}\): \texttt{0.1,\ 0.25,\ 1.0}
  \item
    \(\lambda_{\text{div}}\): \texttt{0.01,\ 0.1,\ 1.0}
  \item
    \(\lambda_{\text{reg}}\): \texttt{0.0,\ 0.01,\ 0.1,\ 0.25,\ 0.5}
  \item
    Objective: \texttt{full,\ vanilla}
  \end{itemize}
\end{itemize}

\end{tcolorbox}

\begin{tcolorbox}[enhanced jigsaw, toptitle=1mm, colframe=quarto-callout-note-color-frame, toprule=.15mm, colbacktitle=quarto-callout-note-color!10!white, bottomrule=.15mm, coltitle=black, arc=.35mm, opacitybacktitle=0.6, bottomtitle=1mm, title={Note \ref*{nte-pen-final-run-eval}: Evaluation Phase}, opacityback=0, titlerule=0mm, rightrule=.15mm, leftrule=.75mm, breakable, left=2mm, colback=white]

\quartocalloutnte{nte-pen-final-run-eval} 

\begin{itemize}
\tightlist
\item
  Generator Parameters:

  \begin{itemize}
  \tightlist
  \item
    \(\lambda_{\text{egy}}\): \texttt{0.1,\ 0.5,\ 1.0,\ 5.0,\ 10.0}
  \end{itemize}
\end{itemize}

\end{tcolorbox}

\subsubsection{Predictive Performance}\label{predictive-performance-2}

Predictive performance measures for this grid search are shown in
Table~\ref{tbl-acc-pen}.

\begin{longtable}{ccccc}

\caption{\label{tbl-acc-pen}Predictive performance measures by dataset
and objective averaged across training-phase parameters
(Note~\ref{nte-pen-final-run-train}) and evaluation-phase parameters
(Note~\ref{nte-pen-final-run-eval}).}

\tabularnewline

  \toprule
  \textbf{Dataset} & \textbf{Variable} & \textbf{Objective} & \textbf{Mean} & \textbf{Se} \\\midrule
  \endfirsthead
  \toprule
  \textbf{Dataset} & \textbf{Variable} & \textbf{Objective} & \textbf{Mean} & \textbf{Se} \\\midrule
  \endhead
  \bottomrule
  \multicolumn{5}{r}{Continuing table below.}\\
  \bottomrule
  \endfoot
  \endlastfoot
  Circ & Accuracy & Full & 0.99 & 0.01 \\
  Circ & Accuracy & Vanilla & 1.0 & 0.0 \\
  Circ & F1-score & Full & 0.99 & 0.01 \\
  Circ & F1-score & Vanilla & 1.0 & 0.0 \\
  LS & Accuracy & Full & 1.0 & 0.01 \\
  LS & Accuracy & Vanilla & 1.0 & 0.0 \\
  LS & F1-score & Full & 1.0 & 0.01 \\
  LS & F1-score & Vanilla & 1.0 & 0.0 \\
  Moon & Accuracy & Full & 0.99 & 0.04 \\
  Moon & Accuracy & Vanilla & 1.0 & 0.01 \\
  Moon & F1-score & Full & 0.99 & 0.04 \\
  Moon & F1-score & Vanilla & 1.0 & 0.01 \\
  OL & Accuracy & Full & 0.91 & 0.02 \\
  OL & Accuracy & Vanilla & 0.92 & 0.0 \\
  OL & F1-score & Full & 0.91 & 0.02 \\
  OL & F1-score & Vanilla & 0.92 & 0.0 \\\bottomrule

\end{longtable}

\subsubsection{Plausibility}\label{plausibility-1}

The results with respect to the plausibility measure are shown in
Figure~\ref{fig-grid-pen-plaus-circles} to
Figure~\ref{fig-grid-pen-plaus-over}.

\begin{figure}

\centering{

\includegraphics[width=0.8\linewidth,height=\textheight,keepaspectratio]{../../paper/experiments/output/final_run/penalties/mlp/circles/evaluation/results/ce/lambda_adversarial---lambda_energy_reg---lambda_energy_diff---lambda_adversarial---lambda_adversarial/plausibility_distance_from_target.png}

}

\caption{\label{fig-grid-pen-plaus-circles}Average outcomes for the
plausibility measure across hyperparameters. This shows the \% change
from the baseline model for the distance-based implausibility metric
(\textbf{?@eq-impl-dist}). Boxplots indicate the variation across
evaluation runs and test settings (varying parameters for \emph{ECCo}).
Data: Circles.}

\end{figure}%

\begin{figure}

\centering{

\includegraphics[width=0.8\linewidth,height=\textheight,keepaspectratio]{../../paper/experiments/output/final_run/penalties/mlp/lin_sep/evaluation/results/ce/lambda_adversarial---lambda_energy_reg---lambda_energy_diff---lambda_adversarial---lambda_adversarial/plausibility_distance_from_target.png}

}

\caption{\label{fig-grid-pen-plaus-lin_sep}Average outcomes for the
plausibility measure across hyperparameters. This shows the \% change
from the baseline model for the distance-based implausibility metric
(\textbf{?@eq-impl-dist}). Boxplots indicate the variation across
evaluation runs and test settings (varying parameters for \emph{ECCo}).
Data: Linearly Separable.}

\end{figure}%

\begin{figure}

\centering{

\includegraphics[width=0.8\linewidth,height=\textheight,keepaspectratio]{../../paper/experiments/output/final_run/penalties/mlp/moons/evaluation/results/ce/lambda_adversarial---lambda_energy_reg---lambda_energy_diff---lambda_adversarial---lambda_adversarial/plausibility_distance_from_target.png}

}

\caption{\label{fig-grid-pen-plaus-moons}Average outcomes for the
plausibility measure across hyperparameters. This shows the \% change
from the baseline model for the distance-based implausibility metric
(\textbf{?@eq-impl-dist}). Boxplots indicate the variation across
evaluation runs and test settings (varying parameters for \emph{ECCo}).
Data: Moons.}

\end{figure}%

\begin{figure}

\centering{

\includegraphics[width=0.8\linewidth,height=\textheight,keepaspectratio]{../../paper/experiments/output/final_run/penalties/mlp/over/evaluation/results/ce/lambda_adversarial---lambda_energy_reg---lambda_energy_diff---lambda_adversarial---lambda_adversarial/plausibility_distance_from_target.png}

}

\caption{\label{fig-grid-pen-plaus-over}Average outcomes for the
plausibility measure across hyperparameters. This shows the \% change
from the baseline model for the distance-based implausibility metric
(\textbf{?@eq-impl-dist}). Boxplots indicate the variation across
evaluation runs and test settings (varying parameters for \emph{ECCo}).
Data: Overlapping.}

\end{figure}%

\subsubsection{Cost}\label{cost-1}

The results with respect to the cost measure are shown in
Figure~\ref{fig-grid-pen-cost-circles} to
Figure~\ref{fig-grid-pen-cost-over}.

\begin{figure}

\centering{

\includegraphics[width=0.8\linewidth,height=\textheight,keepaspectratio]{../../paper/experiments/output/final_run/penalties/mlp/circles/evaluation/results/ce/lambda_adversarial---lambda_energy_reg---lambda_energy_diff---lambda_adversarial---lambda_adversarial/distance.png}

}

\caption{\label{fig-grid-pen-cost-circles}Average outcomes for the cost
measure across hyperparameters. This shows the \% change from the
baseline model for the distance-based cost metric
(\citeproc{ref-wachter2017counterfactual}{Wachter, Mittelstadt, and
Russell 2017}). Boxplots indicate the variation across evaluation runs
and test settings (varying parameters for \emph{ECCo}). Data: Circles.}

\end{figure}%

\begin{figure}

\centering{

\includegraphics[width=0.8\linewidth,height=\textheight,keepaspectratio]{../../paper/experiments/output/final_run/penalties/mlp/lin_sep/evaluation/results/ce/lambda_adversarial---lambda_energy_reg---lambda_energy_diff---lambda_adversarial---lambda_adversarial/distance.png}

}

\caption{\label{fig-grid-pen-cost-lin_sep}Average outcomes for the cost
measure across hyperparameters. This shows the \% change from the
baseline model for the distance-based cost metric
(\citeproc{ref-wachter2017counterfactual}{Wachter, Mittelstadt, and
Russell 2017}). Boxplots indicate the variation across evaluation runs
and test settings (varying parameters for \emph{ECCo}). Data: Linearly
Separable.}

\end{figure}%

\begin{figure}

\centering{

\includegraphics[width=0.8\linewidth,height=\textheight,keepaspectratio]{../../paper/experiments/output/final_run/penalties/mlp/moons/evaluation/results/ce/lambda_adversarial---lambda_energy_reg---lambda_energy_diff---lambda_adversarial---lambda_adversarial/distance.png}

}

\caption{\label{fig-grid-pen-cost-moons}Average outcomes for the cost
measure across hyperparameters. This shows the \% change from the
baseline model for the distance-based cost metric
(\citeproc{ref-wachter2017counterfactual}{Wachter, Mittelstadt, and
Russell 2017}). Boxplots indicate the variation across evaluation runs
and test settings (varying parameters for \emph{ECCo}). Data: Moons.}

\end{figure}%

\begin{figure}

\centering{

\includegraphics[width=0.8\linewidth,height=\textheight,keepaspectratio]{../../paper/experiments/output/final_run/penalties/mlp/over/evaluation/results/ce/lambda_adversarial---lambda_energy_reg---lambda_energy_diff---lambda_adversarial---lambda_adversarial/distance.png}

}

\caption{\label{fig-grid-pen-cost-over}Average outcomes for the cost
measure across hyperparameters. This shows the \% change from the
baseline model for the distance-based cost metric
(\citeproc{ref-wachter2017counterfactual}{Wachter, Mittelstadt, and
Russell 2017}). Boxplots indicate the variation across evaluation runs
and test settings (varying parameters for \emph{ECCo}). Data:
Overlapping.}

\end{figure}%

\subsection{Other Parameters}\label{sec-app-grid-train}

The hyperparameter grid with other varying training parameters is shown
in Note~\ref{nte-train-final-run-train}. The corresponding evaluation
grid used for these experiments is shown in
Note~\ref{nte-train-final-run-eval}.

\begin{tcolorbox}[enhanced jigsaw, toptitle=1mm, colframe=quarto-callout-note-color-frame, toprule=.15mm, colbacktitle=quarto-callout-note-color!10!white, bottomrule=.15mm, coltitle=black, arc=.35mm, opacitybacktitle=0.6, bottomtitle=1mm, title={Note \ref*{nte-train-final-run-train}: Training Phase}, opacityback=0, titlerule=0mm, rightrule=.15mm, leftrule=.75mm, breakable, left=2mm, colback=white]

\quartocalloutnte{nte-train-final-run-train} 

\begin{itemize}
\tightlist
\item
  Generator: \texttt{ecco,\ generic,\ revise}
\item
  Model: \texttt{mlp}
\item
  Training Parameters:

  \begin{itemize}
  \tightlist
  \item
    Burnin: \texttt{0.0,\ 0.5}
  \item
    No.~Counterfactuals: \texttt{100,\ 1000}
  \item
    No.~Epochs: \texttt{50,\ 100}
  \item
    Objective: \texttt{full,\ vanilla}
  \end{itemize}
\end{itemize}

\end{tcolorbox}

\begin{tcolorbox}[enhanced jigsaw, toptitle=1mm, colframe=quarto-callout-note-color-frame, toprule=.15mm, colbacktitle=quarto-callout-note-color!10!white, bottomrule=.15mm, coltitle=black, arc=.35mm, opacitybacktitle=0.6, bottomtitle=1mm, title={Note \ref*{nte-train-final-run-eval}: Evaluation Phase}, opacityback=0, titlerule=0mm, rightrule=.15mm, leftrule=.75mm, breakable, left=2mm, colback=white]

\quartocalloutnte{nte-train-final-run-eval} 

\begin{itemize}
\tightlist
\item
  Generator Parameters:

  \begin{itemize}
  \tightlist
  \item
    \(\lambda_{\text{egy}}\): \texttt{0.1,\ 0.5,\ 1.0,\ 5.0,\ 10.0}
  \end{itemize}
\end{itemize}

\end{tcolorbox}

\subsubsection{Predictive Performance}\label{predictive-performance-3}

Predictive performance measures for this grid search are shown in
Table~\ref{tbl-acc-train}.

\begin{longtable}{ccccc}

\caption{\label{tbl-acc-train}Predictive performance measures by dataset
and objective averaged across training-phase parameters
(Note~\ref{nte-train-final-run-train}) and evaluation-phase parameters
(Note~\ref{nte-train-final-run-eval}).}

\tabularnewline

  \toprule
  \textbf{Dataset} & \textbf{Variable} & \textbf{Objective} & \textbf{Mean} & \textbf{Se} \\\midrule
  \endfirsthead
  \toprule
  \textbf{Dataset} & \textbf{Variable} & \textbf{Objective} & \textbf{Mean} & \textbf{Se} \\\midrule
  \endhead
  \bottomrule
  \multicolumn{5}{r}{Continuing table below.}\\
  \bottomrule
  \endfoot
  \endlastfoot
  Circ & Accuracy & Full & 0.99 & 0.0 \\
  Circ & Accuracy & Vanilla & 1.0 & 0.0 \\
  Circ & F1-score & Full & 0.99 & 0.0 \\
  Circ & F1-score & Vanilla & 1.0 & 0.0 \\
  LS & Accuracy & Full & 1.0 & 0.0 \\
  LS & Accuracy & Vanilla & 1.0 & 0.0 \\
  LS & F1-score & Full & 1.0 & 0.0 \\
  LS & F1-score & Vanilla & 1.0 & 0.0 \\
  Moon & Accuracy & Full & 1.0 & 0.01 \\
  Moon & Accuracy & Vanilla & 0.99 & 0.02 \\
  Moon & F1-score & Full & 1.0 & 0.01 \\
  Moon & F1-score & Vanilla & 0.99 & 0.02 \\
  OL & Accuracy & Full & 0.91 & 0.01 \\
  OL & Accuracy & Vanilla & 0.92 & 0.0 \\
  OL & F1-score & Full & 0.91 & 0.01 \\
  OL & F1-score & Vanilla & 0.92 & 0.0 \\\bottomrule

\end{longtable}

\subsubsection{Plausibility}\label{plausibility-2}

The results with respect to the plausibility measure are shown in
Figure~\ref{fig-grid-train-plaus-circles} to
Figure~\ref{fig-grid-train-plaus-over}.

\begin{figure}

\centering{

\includegraphics[width=0.8\linewidth,height=\textheight,keepaspectratio]{../../paper/experiments/output/final_run/training_params/mlp/circles/evaluation/results/ce/burnin---nce---nepochs---burnin---burnin/plausibility_distance_from_target.png}

}

\caption{\label{fig-grid-train-plaus-circles}Average outcomes for the
plausibility measure across hyperparameters. This shows the \% change
from the baseline model for the distance-based implausibility metric
(\textbf{?@eq-impl-dist}). Boxplots indicate the variation across
evaluation runs and test settings (varying parameters for \emph{ECCo}).
Data: Circles.}

\end{figure}%

\begin{figure}

\centering{

\includegraphics[width=0.8\linewidth,height=\textheight,keepaspectratio]{../../paper/experiments/output/final_run/training_params/mlp/lin_sep/evaluation/results/ce/burnin---nce---nepochs---burnin---burnin/plausibility_distance_from_target.png}

}

\caption{\label{fig-grid-train-plaus-lin_sep}Average outcomes for the
plausibility measure across hyperparameters. This shows the \% change
from the baseline model for the distance-based implausibility metric
(\textbf{?@eq-impl-dist}). Boxplots indicate the variation across
evaluation runs and test settings (varying parameters for \emph{ECCo}).
Data: Linearly Separable.}

\end{figure}%

\begin{figure}

\centering{

\includegraphics[width=0.8\linewidth,height=\textheight,keepaspectratio]{../../paper/experiments/output/final_run/training_params/mlp/moons/evaluation/results/ce/burnin---nce---nepochs---burnin---burnin/plausibility_distance_from_target.png}

}

\caption{\label{fig-grid-train-plaus-moons}Average outcomes for the
plausibility measure across hyperparameters. This shows the \% change
from the baseline model for the distance-based implausibility metric
(\textbf{?@eq-impl-dist}). Boxplots indicate the variation across
evaluation runs and test settings (varying parameters for \emph{ECCo}).
Data: Moons.}

\end{figure}%

\begin{figure}

\centering{

\includegraphics[width=0.8\linewidth,height=\textheight,keepaspectratio]{../../paper/experiments/output/final_run/training_params/mlp/over/evaluation/results/ce/burnin---nce---nepochs---burnin---burnin/plausibility_distance_from_target.png}

}

\caption{\label{fig-grid-train-plaus-over}Average outcomes for the
plausibility measure across hyperparameters. This shows the \% change
from the baseline model for the distance-based implausibility metric
(\textbf{?@eq-impl-dist}). Boxplots indicate the variation across
evaluation runs and test settings (varying parameters for \emph{ECCo}).
Data: Overlapping.}

\end{figure}%

\subsubsection{Cost}\label{cost-2}

The results with respect to the cost measure are shown in
Figure~\ref{fig-grid-train-cost-circles} to
Figure~\ref{fig-grid-train-cost-over}.

\begin{figure}

\centering{

\includegraphics[width=0.8\linewidth,height=\textheight,keepaspectratio]{../../paper/experiments/output/final_run/training_params/mlp/circles/evaluation/results/ce/burnin---nce---nepochs---burnin---burnin/distance.png}

}

\caption{\label{fig-grid-train-cost-circles}Average outcomes for the
cost measure across hyperparameters. This shows the \% change from the
baseline model for the distance-based cost metric
(\citeproc{ref-wachter2017counterfactual}{Wachter, Mittelstadt, and
Russell 2017}). Boxplots indicate the variation across evaluation runs
and test settings (varying parameters for \emph{ECCo}). Data: Circles.}

\end{figure}%

\begin{figure}

\centering{

\includegraphics[width=0.8\linewidth,height=\textheight,keepaspectratio]{../../paper/experiments/output/final_run/training_params/mlp/lin_sep/evaluation/results/ce/burnin---nce---nepochs---burnin---burnin/distance.png}

}

\caption{\label{fig-grid-train-cost-lin_sep}Average outcomes for the
cost measure across hyperparameters. This shows the \% change from the
baseline model for the distance-based cost metric
(\citeproc{ref-wachter2017counterfactual}{Wachter, Mittelstadt, and
Russell 2017}). Boxplots indicate the variation across evaluation runs
and test settings (varying parameters for \emph{ECCo}). Data: Linearly
Separable.}

\end{figure}%

\begin{figure}

\centering{

\includegraphics[width=0.8\linewidth,height=\textheight,keepaspectratio]{../../paper/experiments/output/final_run/training_params/mlp/moons/evaluation/results/ce/burnin---nce---nepochs---burnin---burnin/distance.png}

}

\caption{\label{fig-grid-train-cost-moons}Average outcomes for the cost
measure across hyperparameters. This shows the \% change from the
baseline model for the distance-based cost metric
(\citeproc{ref-wachter2017counterfactual}{Wachter, Mittelstadt, and
Russell 2017}). Boxplots indicate the variation across evaluation runs
and test settings (varying parameters for \emph{ECCo}). Data: Moons.}

\end{figure}%

\begin{figure}

\centering{

\includegraphics[width=0.8\linewidth,height=\textheight,keepaspectratio]{../../paper/experiments/output/final_run/training_params/mlp/over/evaluation/results/ce/burnin---nce---nepochs---burnin---burnin/distance.png}

}

\caption{\label{fig-grid-train-cost-over}Average outcomes for the cost
measure across hyperparameters. This shows the \% change from the
baseline model for the distance-based cost metric
(\citeproc{ref-wachter2017counterfactual}{Wachter, Mittelstadt, and
Russell 2017}). Boxplots indicate the variation across evaluation runs
and test settings (varying parameters for \emph{ECCo}). Data:
Overlapping.}

\end{figure}%

\FloatBarrier

\section{Tuning Key Parameters}\label{sec-app-tune}

Based on the findings from our initial large grid searches
(Section~\ref{sec-app-grid}), we tune selected hyperparameters for all
datasets: namely, the decision threshold \(\tau\) and the strength of
the energy regularization \(\lambda_{\text{reg}}\). The final
hyperparameter choices for each dataset are presented in
Table~\ref{tbl-final-params} in Section~\ref{sec-app-main}. Detailed
results for each data set are shown in Figure~\ref{fig-tune-plaus-adult}
to Figure~\ref{fig-tune-mat-over}. From Table~\ref{tbl-final-params}, we
notice that the same decision threshold of \(\tau=0.5\) is optimal for
all but on dataset. We attribute this to the fact that a low decision
threshold results in a higher share of mature counterfactuals and hence
more opportunities for the model to learn from examples
(Figure~\ref{fig-tune-mat-adult} to Figure~\ref{fig-tune-mat-over}).
This has played a role in particular for our real-world tabular datasets
and MNIST, which suffered from low levels of maturity for higher
decision thresholds. In cases where maturity is not an issue, as for
\emph{Moons}, higher decision thresholds lead to better outcomes, which
may have to do with the fact that the resulting counterfactuals are more
faithful to the model. Concerning the regularization strength, we find
somewhat high variation across datasets. Most notably, we find that
relatively low levels of regularization are optimal for MNIST. We
hypothesize that this finding may be attributed to the uniform scaling
of all input features (digits).

Finally, to increase the proportion of mature counterfactuals for some
datasets, we have also investigated the effect on the learning rate
\(\eta\) for the counterfactual search and even smaller regularization
strengths for a fixed decision threshold of 0.5
(Figure~\ref{fig-tune_lr-plaus-adult} to
Figure~\ref{fig-tune_lr-plaus-over}). For the given low decision
threshold, we find that the learning rate has no discernable impact on
the proportion of mature counterfactuals
(Figure~\ref{fig-tune_lr-mat-adult} to
Figure~\ref{fig-tune_lr-mat-over}). We do notice, however, that the
results for MNIST are much improved when using a low value
\(\lambda_{\text{reg}}\), the strength for the engery regularization:
plausibility is increased by up to \textasciitilde10\%
(Figure~\ref{fig-tune_lr-plaus-mnist}) and the proportion of mature
counterfactuals reaches 100\%.

One consideration worth exploring is to combine high decision thresholds
with high learning rates, which we have not investigated here.

\subsection{Key Parameters}\label{sec-app-tune-key}

The hyperparameter grid for tuning key parameters is shown in
Note~\ref{nte-tune-train}. The corresponding evaluation grid used for
these experiments is shown in Note~\ref{nte-tune-eval}.

\begin{tcolorbox}[enhanced jigsaw, toptitle=1mm, colframe=quarto-callout-note-color-frame, toprule=.15mm, colbacktitle=quarto-callout-note-color!10!white, bottomrule=.15mm, coltitle=black, arc=.35mm, opacitybacktitle=0.6, bottomtitle=1mm, title={Note \ref*{nte-tune-train}: Training Phase}, opacityback=0, titlerule=0mm, rightrule=.15mm, leftrule=.75mm, breakable, left=2mm, colback=white]

\quartocalloutnte{nte-tune-train} 

\begin{itemize}
\tightlist
\item
  Generator Parameters:

  \begin{itemize}
  \tightlist
  \item
    Decision Threshold: \texttt{0.5,\ 0.75,\ 0.9}
  \end{itemize}
\item
  Model: \texttt{mlp}
\item
  Training Parameters:

  \begin{itemize}
  \tightlist
  \item
    \(\lambda_{\text{reg}}\): \texttt{0.1,\ 0.25,\ 0.5}
  \item
    Objective: \texttt{full,\ vanilla}
  \end{itemize}
\end{itemize}

\end{tcolorbox}

\begin{tcolorbox}[enhanced jigsaw, toptitle=1mm, colframe=quarto-callout-note-color-frame, toprule=.15mm, colbacktitle=quarto-callout-note-color!10!white, bottomrule=.15mm, coltitle=black, arc=.35mm, opacitybacktitle=0.6, bottomtitle=1mm, title={Note \ref*{nte-tune-eval}: Evaluation Phase}, opacityback=0, titlerule=0mm, rightrule=.15mm, leftrule=.75mm, breakable, left=2mm, colback=white]

\quartocalloutnte{nte-tune-eval} 

\begin{itemize}
\tightlist
\item
  Generator Parameters:

  \begin{itemize}
  \tightlist
  \item
    \(\lambda_{\text{egy}}\): \texttt{0.1,\ 0.5,\ 1.0,\ 5.0,\ 10.0}
  \end{itemize}
\end{itemize}

\end{tcolorbox}

\subsubsection{Plausibility}\label{plausibility-3}

The results with respect to the plausibility measure are shown in
Figure~\ref{fig-tune-plaus-adult} to Figure~\ref{fig-tune-plaus-over}.

\begin{figure}

\centering{

\includegraphics[width=1\linewidth,height=\textheight,keepaspectratio]{../../paper/experiments/output/final_run/tune/mlp/adult/evaluation/results/ce/lambda_energy_reg---decision_threshold_exper---lambda_energy_eval---lambda_energy_reg---decision_threshold_exper/plausibility_distance_from_target.png}

}

\caption{\label{fig-tune-plaus-adult}Average outcomes for the
plausibility measure across key hyperparameters. This shows the \%
change from the baseline model for the distance-based implausibility
metric (\textbf{?@eq-impl-dist}). Boxplots indicate the variation across
evaluation runs and test settings (varying parameters for \emph{ECCo}).
Data: Adult.}

\end{figure}%

\begin{figure}

\centering{

\includegraphics[width=1\linewidth,height=\textheight,keepaspectratio]{../../paper/experiments/output/final_run/tune/mlp/cali/evaluation/results/ce/lambda_energy_reg---decision_threshold_exper---lambda_energy_eval---lambda_energy_reg---decision_threshold_exper/plausibility_distance_from_target.png}

}

\caption{\label{fig-tune-plaus-cali}Average outcomes for the
plausibility measure across key hyperparameters. This shows the \%
change from the baseline model for the distance-based implausibility
metric (\textbf{?@eq-impl-dist}). Boxplots indicate the variation across
evaluation runs and test settings (varying parameters for \emph{ECCo}).
Data: California Housing.}

\end{figure}%

\begin{figure}

\centering{

\includegraphics[width=1\linewidth,height=\textheight,keepaspectratio]{../../paper/experiments/output/final_run/tune/mlp/circles/evaluation/results/ce/lambda_energy_reg---decision_threshold_exper---lambda_energy_eval---lambda_energy_reg---decision_threshold_exper/plausibility_distance_from_target.png}

}

\caption{\label{fig-tune-plaus-circles}Average outcomes for the
plausibility measure across key hyperparameters. This shows the \%
change from the baseline model for the distance-based implausibility
metric (\textbf{?@eq-impl-dist}). Boxplots indicate the variation across
evaluation runs and test settings (varying parameters for \emph{ECCo}).
Data: Circles.}

\end{figure}%

\begin{figure}

\centering{

\includegraphics[width=1\linewidth,height=\textheight,keepaspectratio]{../../paper/experiments/output/final_run/tune/mlp/credit/evaluation/results/ce/lambda_energy_reg---decision_threshold_exper---lambda_energy_eval---lambda_energy_reg---decision_threshold_exper/plausibility_distance_from_target.png}

}

\caption{\label{fig-tune-plaus-credit}Average outcomes for the
plausibility measure across key hyperparameters. This shows the \%
change from the baseline model for the distance-based implausibility
metric (\textbf{?@eq-impl-dist}). Boxplots indicate the variation across
evaluation runs and test settings (varying parameters for \emph{ECCo}).
Data: Credit.}

\end{figure}%

\begin{figure}

\centering{

\includegraphics[width=1\linewidth,height=\textheight,keepaspectratio]{../../paper/experiments/output/final_run/tune/mlp/gmsc/evaluation/results/ce/lambda_energy_reg---decision_threshold_exper---lambda_energy_eval---lambda_energy_reg---decision_threshold_exper/plausibility_distance_from_target.png}

}

\caption{\label{fig-tune-plaus-gmsc}Average outcomes for the
plausibility measure across key hyperparameters. This shows the \%
change from the baseline model for the distance-based implausibility
metric (\textbf{?@eq-impl-dist}). Boxplots indicate the variation across
evaluation runs and test settings (varying parameters for \emph{ECCo}).
Data: GMSC.}

\end{figure}%

\begin{figure}

\centering{

\includegraphics[width=1\linewidth,height=\textheight,keepaspectratio]{../../paper/experiments/output/final_run/tune/mlp/lin_sep/evaluation/results/ce/lambda_energy_reg---decision_threshold_exper---lambda_energy_eval---lambda_energy_reg---decision_threshold_exper/plausibility_distance_from_target.png}

}

\caption{\label{fig-tune-plaus-lin_sep}Average outcomes for the
plausibility measure across key hyperparameters. This shows the \%
change from the baseline model for the distance-based implausibility
metric (\textbf{?@eq-impl-dist}). Boxplots indicate the variation across
evaluation runs and test settings (varying parameters for \emph{ECCo}).
Data: Linearly Separable.}

\end{figure}%

\begin{figure}

\centering{

\includegraphics[width=1\linewidth,height=\textheight,keepaspectratio]{../../paper/experiments/output/final_run/tune/mlp/mnist/evaluation/results/ce/lambda_energy_reg---decision_threshold_exper---lambda_energy_eval---lambda_energy_reg---decision_threshold_exper/plausibility_distance_from_target.png}

}

\caption{\label{fig-tune-plaus-mnist}Average outcomes for the
plausibility measure across key hyperparameters. This shows the \%
change from the baseline model for the distance-based implausibility
metric (\textbf{?@eq-impl-dist}). Boxplots indicate the variation across
evaluation runs and test settings (varying parameters for \emph{ECCo}).
Data: MNIST.}

\end{figure}%

\begin{figure}

\centering{

\includegraphics[width=1\linewidth,height=\textheight,keepaspectratio]{../../paper/experiments/output/final_run/tune/mlp/moons/evaluation/results/ce/lambda_energy_reg---decision_threshold_exper---lambda_energy_eval---lambda_energy_reg---decision_threshold_exper/plausibility_distance_from_target.png}

}

\caption{\label{fig-tune-plaus-moons}Average outcomes for the
plausibility measure across key hyperparameters. This shows the \%
change from the baseline model for the distance-based implausibility
metric (\textbf{?@eq-impl-dist}). Boxplots indicate the variation across
evaluation runs and test settings (varying parameters for \emph{ECCo}).
Data: Moons.}

\end{figure}%

\begin{figure}

\centering{

\includegraphics[width=1\linewidth,height=\textheight,keepaspectratio]{../../paper/experiments/output/final_run/tune/mlp/over/evaluation/results/ce/lambda_energy_reg---decision_threshold_exper---lambda_energy_eval---lambda_energy_reg---decision_threshold_exper/plausibility_distance_from_target.png}

}

\caption{\label{fig-tune-plaus-over}Average outcomes for the
plausibility measure across key hyperparameters. This shows the \%
change from the baseline model for the distance-based implausibility
metric (\textbf{?@eq-impl-dist}). Boxplots indicate the variation across
evaluation runs and test settings (varying parameters for \emph{ECCo}).
Data: Overlapping.}

\end{figure}%

\subsubsection{Proportion of Mature CE}\label{proportion-of-mature-ce}

The results with respect to the proportion of mature counterfactuals in
each epoch are shown in Figure~\ref{fig-tune-mat-adult} to
Figure~\ref{fig-tune-mat-over}.

\begin{figure}

\centering{

\includegraphics[width=1\linewidth,height=\textheight,keepaspectratio]{../../paper/experiments/output/final_run/tune/mlp/adult/evaluation/results/logs/objective---decision_threshold---lambda_energy_reg/percent_valid.png}

}

\caption{\label{fig-tune-mat-adult}Proportion of mature counterfactuals
in each epoch. Data: Adult.}

\end{figure}%

\begin{figure}

\centering{

\includegraphics[width=1\linewidth,height=\textheight,keepaspectratio]{../../paper/experiments/output/final_run/tune/mlp/cali/evaluation/results/logs/objective---decision_threshold---lambda_energy_reg/percent_valid.png}

}

\caption{\label{fig-tune-mat-cali}Proportion of mature counterfactuals
in each epoch. Data: California Housing.}

\end{figure}%

\begin{figure}

\centering{

\includegraphics[width=1\linewidth,height=\textheight,keepaspectratio]{../../paper/experiments/output/final_run/tune/mlp/circles/evaluation/results/logs/objective---decision_threshold---lambda_energy_reg/percent_valid.png}

}

\caption{\label{fig-tune-mat-circles}Proportion of mature
counterfactuals in each epoch. Data: Circles.}

\end{figure}%

\begin{figure}

\centering{

\includegraphics[width=1\linewidth,height=\textheight,keepaspectratio]{../../paper/experiments/output/final_run/tune/mlp/credit/evaluation/results/logs/objective---decision_threshold---lambda_energy_reg/percent_valid.png}

}

\caption{\label{fig-tune-mat-credit}Proportion of mature counterfactuals
in each epoch. Data: Credit.}

\end{figure}%

\begin{figure}

\centering{

\includegraphics[width=1\linewidth,height=\textheight,keepaspectratio]{../../paper/experiments/output/final_run/tune/mlp/gmsc/evaluation/results/logs/objective---decision_threshold---lambda_energy_reg/percent_valid.png}

}

\caption{\label{fig-tune-mat-gmsc}Proportion of mature counterfactuals
in each epoch. Data: GMSC.}

\end{figure}%

\begin{figure}

\centering{

\includegraphics[width=1\linewidth,height=\textheight,keepaspectratio]{../../paper/experiments/output/final_run/tune/mlp/lin_sep/evaluation/results/logs/objective---decision_threshold---lambda_energy_reg/percent_valid.png}

}

\caption{\label{fig-tune-mat-lin_sep}Proportion of mature
counterfactuals in each epoch. Data: Linearly Separable.}

\end{figure}%

\begin{figure}

\centering{

\includegraphics[width=1\linewidth,height=\textheight,keepaspectratio]{../../paper/experiments/output/final_run/tune/mlp/mnist/evaluation/results/logs/objective---decision_threshold---lambda_energy_reg/percent_valid.png}

}

\caption{\label{fig-tune-mat-mnist}Proportion of mature counterfactuals
in each epoch. Data: MNIST.}

\end{figure}%

\begin{figure}

\centering{

\includegraphics[width=1\linewidth,height=\textheight,keepaspectratio]{../../paper/experiments/output/final_run/tune/mlp/moons/evaluation/results/logs/objective---decision_threshold---lambda_energy_reg/percent_valid.png}

}

\caption{\label{fig-tune-mat-moons}Proportion of mature counterfactuals
in each epoch. Data: Moons.}

\end{figure}%

\begin{figure}

\centering{

\includegraphics[width=1\linewidth,height=\textheight,keepaspectratio]{../../paper/experiments/output/final_run/tune/mlp/over/evaluation/results/logs/objective---decision_threshold---lambda_energy_reg/percent_valid.png}

}

\caption{\label{fig-tune-mat-over}Proportion of mature counterfactuals
in each epoch. Data: Overlapping.}

\end{figure}%

\subsection{Learning Rate}\label{sec-app-tune-lr}

The hyperparameter grid for tuning the learning rate is shown in
Note~\ref{nte-tune_lr-train}. The corresponding evaluation grid used for
these experiments is shown in Note~\ref{nte-tune_lr-eval}.

\begin{tcolorbox}[enhanced jigsaw, toptitle=1mm, colframe=quarto-callout-note-color-frame, toprule=.15mm, colbacktitle=quarto-callout-note-color!10!white, bottomrule=.15mm, coltitle=black, arc=.35mm, opacitybacktitle=0.6, bottomtitle=1mm, title={Note \ref*{nte-tune_lr-train}: Training Phase}, opacityback=0, titlerule=0mm, rightrule=.15mm, leftrule=.75mm, breakable, left=2mm, colback=white]

\quartocalloutnte{nte-tune_lr-train} 

\begin{itemize}
\tightlist
\item
  Generator Parameters:

  \begin{itemize}
  \tightlist
  \item
    Learning Rate: \texttt{0.1,\ 0.5,\ 1.0}
  \end{itemize}
\item
  Model: \texttt{mlp}
\item
  Training Parameters:

  \begin{itemize}
  \tightlist
  \item
    \(\lambda_{\text{reg}}\): \texttt{0.01,\ 0.1,\ 0.5}
  \item
    Objective: \texttt{full,\ vanilla}
  \end{itemize}
\end{itemize}

\end{tcolorbox}

\begin{tcolorbox}[enhanced jigsaw, toptitle=1mm, colframe=quarto-callout-note-color-frame, toprule=.15mm, colbacktitle=quarto-callout-note-color!10!white, bottomrule=.15mm, coltitle=black, arc=.35mm, opacitybacktitle=0.6, bottomtitle=1mm, title={Note \ref*{nte-tune_lr-eval}: Evaluation Phase}, opacityback=0, titlerule=0mm, rightrule=.15mm, leftrule=.75mm, breakable, left=2mm, colback=white]

\quartocalloutnte{nte-tune_lr-eval} 

\begin{itemize}
\tightlist
\item
  Generator Parameters:

  \begin{itemize}
  \tightlist
  \item
    \(\lambda_{\text{egy}}\): \texttt{0.1,\ 0.5,\ 1.0,\ 5.0,\ 10.0}
  \end{itemize}
\end{itemize}

\end{tcolorbox}

\subsubsection{Plausibility}\label{plausibility-4}

The results with respect to the plausibility measure are shown in
Figure~\ref{fig-tune_lr-plaus-adult} to
Figure~\ref{fig-tune_lr-plaus-over}.

\begin{figure}

\centering{

\includegraphics[width=1\linewidth,height=\textheight,keepaspectratio]{../../paper/experiments/output/final_run/tune_lr/mlp/adult/evaluation/results/ce/lambda_energy_reg---lr_exper---lambda_energy_eval---lambda_energy_reg---lr_exper/plausibility_distance_from_target.png}

}

\caption{\label{fig-tune_lr-plaus-adult}Average outcomes for the
plausibility measure across key hyperparameters. This shows the \%
change from the baseline model for the distance-based implausibility
metric (\textbf{?@eq-impl-dist}). Boxplots indicate the variation across
evaluation runs and test settings (varying parameters for \emph{ECCo}).
Data: Adult.}

\end{figure}%

\begin{figure}

\centering{

\includegraphics[width=1\linewidth,height=\textheight,keepaspectratio]{../../paper/experiments/output/final_run/tune_lr/mlp/cali/evaluation/results/ce/lambda_energy_reg---lr_exper---lambda_energy_eval---lambda_energy_reg---lr_exper/plausibility_distance_from_target.png}

}

\caption{\label{fig-tune_lr-plaus-cali}Average outcomes for the
plausibility measure across key hyperparameters. This shows the \%
change from the baseline model for the distance-based implausibility
metric (\textbf{?@eq-impl-dist}). Boxplots indicate the variation across
evaluation runs and test settings (varying parameters for \emph{ECCo}).
Data: California Housing.}

\end{figure}%

\begin{figure}

\centering{

\includegraphics[width=1\linewidth,height=\textheight,keepaspectratio]{../../paper/experiments/output/final_run/tune_lr/mlp/circles/evaluation/results/ce/lambda_energy_reg---lr_exper---lambda_energy_eval---lambda_energy_reg---lr_exper/plausibility_distance_from_target.png}

}

\caption{\label{fig-tune_lr-plaus-circles}Average outcomes for the
plausibility measure across key hyperparameters. This shows the \%
change from the baseline model for the distance-based implausibility
metric (\textbf{?@eq-impl-dist}). Boxplots indicate the variation across
evaluation runs and test settings (varying parameters for \emph{ECCo}).
Data: Circles.}

\end{figure}%

\begin{figure}

\centering{

\includegraphics[width=1\linewidth,height=\textheight,keepaspectratio]{../../paper/experiments/output/final_run/tune_lr/mlp/credit/evaluation/results/ce/lambda_energy_reg---lr_exper---lambda_energy_eval---lambda_energy_reg---lr_exper/plausibility_distance_from_target.png}

}

\caption{\label{fig-tune_lr-plaus-credit}Average outcomes for the
plausibility measure across key hyperparameters. This shows the \%
change from the baseline model for the distance-based implausibility
metric (\textbf{?@eq-impl-dist}). Boxplots indicate the variation across
evaluation runs and test settings (varying parameters for \emph{ECCo}).
Data: Credit.}

\end{figure}%

\begin{figure}

\centering{

\includegraphics[width=1\linewidth,height=\textheight,keepaspectratio]{../../paper/experiments/output/final_run/tune_lr/mlp/gmsc/evaluation/results/ce/lambda_energy_reg---lr_exper---lambda_energy_eval---lambda_energy_reg---lr_exper/plausibility_distance_from_target.png}

}

\caption{\label{fig-tune_lr-plaus-gmsc}Average outcomes for the
plausibility measure across key hyperparameters. This shows the \%
change from the baseline model for the distance-based implausibility
metric (\textbf{?@eq-impl-dist}). Boxplots indicate the variation across
evaluation runs and test settings (varying parameters for \emph{ECCo}).
Data: GMSC.}

\end{figure}%

\begin{figure}

\centering{

\includegraphics[width=1\linewidth,height=\textheight,keepaspectratio]{../../paper/experiments/output/final_run/tune_lr/mlp/lin_sep/evaluation/results/ce/lambda_energy_reg---lr_exper---lambda_energy_eval---lambda_energy_reg---lr_exper/plausibility_distance_from_target.png}

}

\caption{\label{fig-tune_lr-plaus-lin_sep}Average outcomes for the
plausibility measure across key hyperparameters. This shows the \%
change from the baseline model for the distance-based implausibility
metric (\textbf{?@eq-impl-dist}). Boxplots indicate the variation across
evaluation runs and test settings (varying parameters for \emph{ECCo}).
Data: Linearly Separable.}

\end{figure}%

\begin{figure}

\centering{

\includegraphics[width=1\linewidth,height=\textheight,keepaspectratio]{../../paper/experiments/output/final_run/tune_lr/mlp/mnist/evaluation/results/ce/lambda_energy_reg---lr_exper---lambda_energy_eval---lambda_energy_reg---lr_exper/plausibility_distance_from_target.png}

}

\caption{\label{fig-tune_lr-plaus-mnist}Average outcomes for the
plausibility measure across key hyperparameters. This shows the \%
change from the baseline model for the distance-based implausibility
metric (\textbf{?@eq-impl-dist}). Boxplots indicate the variation across
evaluation runs and test settings (varying parameters for \emph{ECCo}).
Data: MNIST.}

\end{figure}%

\begin{figure}

\centering{

\includegraphics[width=1\linewidth,height=\textheight,keepaspectratio]{../../paper/experiments/output/final_run/tune_lr/mlp/moons/evaluation/results/ce/lambda_energy_reg---lr_exper---lambda_energy_eval---lambda_energy_reg---lr_exper/plausibility_distance_from_target.png}

}

\caption{\label{fig-tune_lr-plaus-moons}Average outcomes for the
plausibility measure across key hyperparameters. This shows the \%
change from the baseline model for the distance-based implausibility
metric (\textbf{?@eq-impl-dist}). Boxplots indicate the variation across
evaluation runs and test settings (varying parameters for \emph{ECCo}).
Data: Moons.}

\end{figure}%

\begin{figure}

\centering{

\includegraphics[width=1\linewidth,height=\textheight,keepaspectratio]{../../paper/experiments/output/final_run/tune_lr/mlp/over/evaluation/results/ce/lambda_energy_reg---lr_exper---lambda_energy_eval---lambda_energy_reg---lr_exper/plausibility_distance_from_target.png}

}

\caption{\label{fig-tune_lr-plaus-over}Average outcomes for the
plausibility measure across key hyperparameters. This shows the \%
change from the baseline model for the distance-based implausibility
metric (\textbf{?@eq-impl-dist}). Boxplots indicate the variation across
evaluation runs and test settings (varying parameters for \emph{ECCo}).
Data: Overlapping.}

\end{figure}%

\subsubsection{Proportion of Mature CE}\label{proportion-of-mature-ce-1}

The results with respect to the proportion of mature counterfactuals in
each epoch are shown in Figure~\ref{fig-tune_lr-mat-adult} to
Figure~\ref{fig-tune_lr-mat-over}.

\begin{figure}

\centering{

\includegraphics[width=1\linewidth,height=\textheight,keepaspectratio]{../../paper/experiments/output/final_run/tune_lr/mlp/adult/evaluation/results/logs/objective---lr---lambda_energy_reg/percent_valid.png}

}

\caption{\label{fig-tune_lr-mat-adult}Proportion of mature
counterfactuals in each epoch. Data: Adult.}

\end{figure}%

\begin{figure}

\centering{

\includegraphics[width=1\linewidth,height=\textheight,keepaspectratio]{../../paper/experiments/output/final_run/tune_lr/mlp/cali/evaluation/results/logs/objective---lr---lambda_energy_reg/percent_valid.png}

}

\caption{\label{fig-tune_lr-mat-cali}Proportion of mature
counterfactuals in each epoch. Data: California Housing.}

\end{figure}%

\begin{figure}

\centering{

\includegraphics[width=1\linewidth,height=\textheight,keepaspectratio]{../../paper/experiments/output/final_run/tune_lr/mlp/circles/evaluation/results/logs/objective---lr---lambda_energy_reg/percent_valid.png}

}

\caption{\label{fig-tune_lr-mat-circles}Proportion of mature
counterfactuals in each epoch. Data: Circles.}

\end{figure}%

\begin{figure}

\centering{

\includegraphics[width=1\linewidth,height=\textheight,keepaspectratio]{../../paper/experiments/output/final_run/tune_lr/mlp/credit/evaluation/results/logs/objective---lr---lambda_energy_reg/percent_valid.png}

}

\caption{\label{fig-tune_lr-mat-credit}Proportion of mature
counterfactuals in each epoch. Data: Credit.}

\end{figure}%

\begin{figure}

\centering{

\includegraphics[width=1\linewidth,height=\textheight,keepaspectratio]{../../paper/experiments/output/final_run/tune_lr/mlp/gmsc/evaluation/results/logs/objective---lr---lambda_energy_reg/percent_valid.png}

}

\caption{\label{fig-tune_lr-mat-gmsc}Proportion of mature
counterfactuals in each epoch. Data: GMSC.}

\end{figure}%

\begin{figure}

\centering{

\includegraphics[width=1\linewidth,height=\textheight,keepaspectratio]{../../paper/experiments/output/final_run/tune_lr/mlp/lin_sep/evaluation/results/logs/objective---lr---lambda_energy_reg/percent_valid.png}

}

\caption{\label{fig-tune_lr-mat-lin_sep}Proportion of mature
counterfactuals in each epoch. Data: Linearly Separable.}

\end{figure}%

\begin{figure}

\centering{

\includegraphics[width=1\linewidth,height=\textheight,keepaspectratio]{../../paper/experiments/output/final_run/tune_lr/mlp/mnist/evaluation/results/logs/objective---lr---lambda_energy_reg/percent_valid.png}

}

\caption{\label{fig-tune_lr-mat-mnist}Proportion of mature
counterfactuals in each epoch. Data: MNIST.}

\end{figure}%

\begin{figure}

\centering{

\includegraphics[width=1\linewidth,height=\textheight,keepaspectratio]{../../paper/experiments/output/final_run/tune_lr/mlp/moons/evaluation/results/logs/objective---lr---lambda_energy_reg/percent_valid.png}

}

\caption{\label{fig-tune_lr-mat-moons}Proportion of mature
counterfactuals in each epoch. Data: Moons.}

\end{figure}%

\begin{figure}

\centering{

\includegraphics[width=1\linewidth,height=\textheight,keepaspectratio]{../../paper/experiments/output/final_run/tune_lr/mlp/over/evaluation/results/logs/objective---lr---lambda_energy_reg/percent_valid.png}

}

\caption{\label{fig-tune_lr-mat-over}Proportion of mature
counterfactuals in each epoch. Data: Overlapping.}

\end{figure}%

\FloatBarrier

\section{Computation Details}\label{computation-details}

\subsection{Hardware}\label{sec-app-hardware}

We performed our experiments on a high-performance cluster. Details
about the cluster will be disclosed upon publication to avoid revealing
information that might interfere with the double-blind review process.
Since our experiments involve highly parallel tasks and rather small
models by today's standard, we have relied on distributed computing
across multiple central processing units (CPU). Graphical processing
units (GPU) were not required.

\subsubsection{Grid Searches}\label{grid-searches}

Model training for the largest grid searches with 270 unique parameter
combinations was parallelized across 34 CPUs with 2GB memory each. The
time to completion varied by dataset for reasons discussed in
\textbf{?@sec-discussion}: 0h49m (\emph{Moons}), 1h4m (\emph{Linearly
Separable}), 1h49m (\emph{Circles}), 3h52m (\emph{Overlapping}). Model
evaluations for large grid searches were parallelized across 20 CPUs
with 3GB memory each. Evaluations for all data sets took less than one
hour (\textless1h) to complete.

\subsubsection{Tuning}\label{tuning}

For tuning of selected hyperparameters, we distributed the task of
generating counterfactuals during training across 40 CPUs with 2GB
memory each for all tabular datasets. Except for the \emph{Adult}
dataset, all training runs were completed in less that half an hour
(\textless0h30m). The \emph{Adult} dataset took around 0h35m to
complete. Evaluations across 20 CPUs with 3GB memory each generally took
less than 0h30m to complete. For \emph{MNIST}, we relied on 100 CPUs
with 2GB memory each. For the \emph{MLP}, training of all models could
be completed in 1h30m, while the evaluation across 20 CPUs (6GB memory)
took 4h12m. For the \emph{CNN}, training of all models took
\textasciitilde8h, with conventionally trained models taking
\textasciitilde0h15m each and model with CT taking
\textasciitilde0h30m-0h45m each.

\subsection{Software}\label{software}

All computations were performed in the Julia Programming Language
(\citeproc{ref-bezanson2017julia}{Bezanson et al. 2017}). We have
developed a package for counterfactual training that leverages and
extends the functionality provided by several existing packages, most
notably
\href{https://github.com/JuliaTrustworthyAI/CounterfactualExplanations.jl}{CounterfactualExplanations.jl}
(\citeproc{ref-altmeyer2023explaining}{Altmeyer, Deursen, and Liem
2023}) and the \href{https://fluxml.ai/Flux.jl/v0.16/}{Flux.jl} library
for deep learning (\citeproc{ref-innes2018fashionable}{Michael Innes et
al. 2018}; \citeproc{ref-innes2018flux}{Mike Innes 2018}). For
data-wrangling and presentation-ready tables we relied on
\href{https://dataframes.juliadata.org/v1.7/}{DataFrames.jl}
(\citeproc{ref-milan2023dataframes}{Bouchet-Valat and Kamiński 2023})
and
\href{https://ronisbr.github.io/PrettyTables.jl/v2.4/}{PrettyTables.jl}
(\citeproc{ref-chagas2024pretty}{Chagas et al. 2024}), respectively. For
plots and visualizations we used both
\href{https://docs.juliaplots.org/v1.40/}{Plots.jl}
(\citeproc{ref-PlotsJL}{Christ et al. 2023}) and
\href{https://docs.makie.org/v0.22/}{Makie.jl}
(\citeproc{ref-danisch2021makie}{Danisch and Krumbiegel 2021}), in
particular \href{https://aog.makie.org/v0.9.3/}{AlgebraOfGraphics.jl}.
To distribute computational tasks across multiple processors, we have
relied on \href{https://juliaparallel.org/MPI.jl/v0.20/}{MPI.jl}
(\citeproc{ref-byrne2021mpi}{Byrne, Wilcox, and Churavy 2021}).

\end{appendices}

\phantomsection\label{refs}
\begin{CSLReferences}{1}{0}
\bibitem[\citeproctext]{ref-altmeyer2023explaining}
Altmeyer, Patrick, Arie van Deursen, and Cynthia C. S. Liem. 2023.
{``{Explaining Black-Box Models through Counterfactuals}.''} In
\emph{Proceedings of the JuliaCon Conferences}, 1:130.

\bibitem[\citeproctext]{ref-altmeyer2024faithful}
Altmeyer, Patrick, Mojtaba Farmanbar, Arie van Deursen, and Cynthia C.
S. Liem. 2024. {``{Faithful Model Explanations through
Energy-Constrained Conformal Counterfactuals}.''} In \emph{Proceedings
of the Thirty-Eighth AAAI Conference on Artificial Intelligence},
38:10829--37. 10. \url{https://doi.org/10.1609/aaai.v38i10.28956}.

\bibitem[\citeproctext]{ref-bezanson2017julia}
Bezanson, Jeff, Alan Edelman, Stefan Karpinski, and Viral B Shah. 2017.
{``Julia: A Fresh Approach to Numerical Computing.''} \emph{SIAM Review}
59 (1): 65--98. \url{https://doi.org/10.1137/141000671}.

\bibitem[\citeproctext]{ref-milan2023dataframes}
Bouchet-Valat, Milan, and Bogumił Kamiński. 2023. {``DataFrames.jl:
Flexible and Fast Tabular Data in Julia.''} \emph{Journal of Statistical
Software} 107 (4): 1--32. \url{https://doi.org/10.18637/jss.v107.i04}.

\bibitem[\citeproctext]{ref-byrne2021mpi}
Byrne, Simon, Lucas C. Wilcox, and Valentin Churavy. 2021. {``MPI.jl:
Julia Bindings for the Message Passing Interface.''} \emph{Proceedings
of the JuliaCon Conferences} 1 (1): 68.
\url{https://doi.org/10.21105/jcon.00068}.

\bibitem[\citeproctext]{ref-chagas2024pretty}
Chagas, Ronan Arraes Jardim, Ben Baumgold, Glen Hertz, Hendrik Ranocha,
Mark Wells, Nathan Boyer, Nicholas Ritchie, et al. 2024.
{``Ronisbr/PrettyTables.jl: V2.4.0.''} Zenodo.
\url{https://doi.org/10.5281/zenodo.13835553}.

\bibitem[\citeproctext]{ref-PlotsJL}
Christ, Simon, Daniel Schwabeneder, Christopher Rackauckas, Michael
Krabbe Borregaard, and Thomas Breloff. 2023. {``Plots.jl -- a User
Extendable Plotting API for the Julia Programming Language.''}
https://doi.org/\url{https://doi.org/10.5334/jors.431}.

\bibitem[\citeproctext]{ref-danisch2021makie}
Danisch, Simon, and Julius Krumbiegel. 2021. {``{Makie.jl}: Flexible
High-Performance Data Visualization for {Julia}.''} \emph{Journal of
Open Source Software} 6 (65): 3349.
\url{https://doi.org/10.21105/joss.03349}.

\bibitem[\citeproctext]{ref-goodfellow2014explaining}
Goodfellow, Ian, Jonathon Shlens, and Christian Szegedy. 2015.
{``{Explaining and Harnessing Adversarial Examples}.''}
\url{https://arxiv.org/abs/1412.6572}.

\bibitem[\citeproctext]{ref-grathwohl2020your}
Grathwohl, Will, Kuan-Chieh Wang, Joern-Henrik Jacobsen, David Duvenaud,
Mohammad Norouzi, and Kevin Swersky. 2020. {``Your Classifier Is
Secretly an Energy Based Model and You Should Treat It Like One.''} In
\emph{International Conference on Learning Representations}.

\bibitem[\citeproctext]{ref-gretton2012kernel}
Gretton, Arthur, Karsten M Borgwardt, Malte J Rasch, Bernhard Schölkopf,
and Alexander Smola. 2012. {``A Kernel Two-Sample Test.''} \emph{The
Journal of Machine Learning Research} 13 (1): 723--73.

\bibitem[\citeproctext]{ref-hastie2009elements}
Hastie, Trevor, Robert Tibshirani, and Jerome Friedman. 2009. \emph{The
Elements of Statistical Learning}. Springer New York.
\url{https://doi.org/10.1007/978-0-387-84858-7}.

\bibitem[\citeproctext]{ref-innes2018fashionable}
Innes, Michael, Elliot Saba, Keno Fischer, Dhairya Gandhi, Marco
Concetto Rudilosso, Neethu Mariya Joy, Tejan Karmali, Avik Pal, and
Viral Shah. 2018. {``Fashionable Modelling with Flux.''}
\url{https://arxiv.org/abs/1811.01457}.

\bibitem[\citeproctext]{ref-innes2018flux}
Innes, Mike. 2018. {``Flux: {Elegant} Machine Learning with {Julia}.''}
\emph{Journal of Open Source Software} 3 (25): 602.
\url{https://doi.org/10.21105/joss.00602}.

\bibitem[\citeproctext]{ref-joshi2019realistic}
Joshi, Shalmali, Oluwasanmi Koyejo, Warut Vijitbenjaronk, Been Kim, and
Joydeep Ghosh. 2019. {``{Towards Realistic Individual Recourse and
Actionable Explanations in Black-Box Decision Making Systems}.''}
\url{https://arxiv.org/abs/1907.09615}.

\bibitem[\citeproctext]{ref-wachter2017counterfactual}
Wachter, Sandra, Brent Mittelstadt, and Chris Russell. 2017.
{``Counterfactual Explanations Without Opening the Black Box:
{Automated} Decisions and the {GDPR}.''} \emph{Harv. JL \& Tech.} 31:
841. \url{https://doi.org/10.2139/ssrn.3063289}.

\end{CSLReferences}




\end{document}
