%%%%%%%% ICML 2025 EXAMPLE LATEX SUBMISSION FILE %%%%%%%%%%%%%%%%%

\documentclass{article}

% Recommended, but optional, packages for figures and better typesetting:
\usepackage{microtype}
\usepackage{graphicx}
\usepackage{subfigure}
\usepackage{booktabs} % for professional tables

% hyperref makes hyperlinks in the resulting PDF.
% If your build breaks (sometimes temporarily if a hyperlink spans a page)
% please comment out the following usepackage line and replace
% \usepackage{icml2025} with \usepackage[nohyperref]{icml2025} above.
\usepackage{hyperref}


% Attempt to make hyperref and algorithmic work together better:
\newcommand{\theHalgorithm}{\arabic{algorithm}}

% Use the following line for the initial blind version submitted for review:
% \usepackage{icml2025}

% If accepted, instead use the following line for the camera-ready submission:
\usepackage[accepted]{icml2025}

% For theorems and such
\usepackage{amsmath}
\usepackage{amssymb}
\usepackage{mathtools}
\usepackage{amsthm}

% if you use cleveref..
\usepackage[capitalize,noabbrev]{cleveref}

%%%%%%%%%%%%%%%%%%%%%%%%%%%%%%%%
% THEOREMS
%%%%%%%%%%%%%%%%%%%%%%%%%%%%%%%%
\theoremstyle{plain}
\newtheorem{theorem}{Theorem}[section]
\newtheorem{proposition}[theorem]{Proposition}
\newtheorem{lemma}[theorem]{Lemma}
\newtheorem{corollary}[theorem]{Corollary}
\theoremstyle{definition}
\newtheorem{definition}[theorem]{Definition}
\newtheorem{assumption}[theorem]{Assumption}
\theoremstyle{remark}
\newtheorem{remark}[theorem]{Remark}

% Todonotes is useful during development; simply uncomment the next line
%    and comment out the line below the next line to turn off comments
%\usepackage[disable,textsize=tiny]{todonotes}
\usepackage[textsize=tiny]{todonotes}


% The \icmltitle you define below is probably too long as a header.
% Therefore, a short form for the running title is supplied here:
\icmltitlerunning{Submission and Formatting Instructions for ICML 2025}

\begin{document}

\twocolumn[
\icmltitle{Submission and Formatting Instructions for \\
           International Conference on Machine Learning (ICML 2025)}

% It is OKAY to include author information, even for blind
% submissions: the style file will automatically remove it for you
% unless you've provided the [accepted] option to the icml2025
% package.

% List of affiliations: The first argument should be a (short)
% identifier you will use later to specify author affiliations
% Academic affiliations should list Department, University, City, Region, Country
% Industry affiliations should list Company, City, Region, Country

% You can specify symbols, otherwise they are numbered in order.
% Ideally, you should not use this facility. Affiliations will be numbered
% in order of appearance and this is the preferred way.
\icmlsetsymbol{equal}{*}

\begin{icmlauthorlist}
\icmlauthor{Firstname1 Lastname1}{equal,yyy}
\icmlauthor{Firstname2 Lastname2}{equal,yyy,comp}
\icmlauthor{Firstname3 Lastname3}{comp}
\icmlauthor{Firstname4 Lastname4}{sch}
\icmlauthor{Firstname5 Lastname5}{yyy}
\icmlauthor{Firstname6 Lastname6}{sch,yyy,comp}
\icmlauthor{Firstname7 Lastname7}{comp}
%\icmlauthor{}{sch}
\icmlauthor{Firstname8 Lastname8}{sch}
\icmlauthor{Firstname8 Lastname8}{yyy,comp}
%\icmlauthor{}{sch}
%\icmlauthor{}{sch}
\end{icmlauthorlist}

\icmlaffiliation{yyy}{Department of XXX, University of YYY, Location, Country}
\icmlaffiliation{comp}{Company Name, Location, Country}
\icmlaffiliation{sch}{School of ZZZ, Institute of WWW, Location, Country}

\icmlcorrespondingauthor{Firstname1 Lastname1}{first1.last1@xxx.edu}
\icmlcorrespondingauthor{Firstname2 Lastname2}{first2.last2@www.uk}

% You may provide any keywords that you
% find helpful for describing your paper; these are used to populate
% the "keywords" metadata in the PDF but will not be shown in the document
\icmlkeywords{Machine Learning, ICML}

\vskip 0.3in
]

% this must go after the closing bracket ] following \twocolumn[ ...

% This command actually creates the footnote in the first column
% listing the affiliations and the copyright notice.
% The command takes one argument, which is text to display at the start of the footnote.
% The \icmlEqualContribution command is standard text for equal contribution.
% Remove it (just {}) if you do not need this facility.

%\printAffiliationsAndNotice{}  % leave blank if no need to mention equal contribution
\printAffiliationsAndNotice{\icmlEqualContribution} % otherwise use the standard text.

\begin{abstract}
This document provides a basic paper template and submission guidelines.
Abstracts must be a single paragraph, ideally between 4--6 sentences long.
Gross violations will trigger corrections at the camera-ready phase.

\keywords{First keyword  \and Second keyword \and Another keyword.}
\end{abstract}

\section{Related Literature} \label{sec-lit}

\subsection{Background on Counterfactual Explanations}

\subsection{Learning Representations}

\textcolor{red}{For example, joint-energy models ...}

\subsection{Generalization and Robustness}

\cite{sauer2021counterfactual} generate counterfactual images for MNIST and ImageNet through independent mechanisms (IM): each IM learns class-conditional input distributions over a specific lower-dimensional, semantically meaningful factor, such as \textit{texture}, \textit{shape} and \textit{background}. The demonstrate that using these generated counterfactuals during classifier training improves model robustness. Similarly, \cite{abbasnejad2020counterfactual} argue that counterfactuals represent potentially useful training data in machine learning, especially in supervised settings where inputs may be reasonably mapped to multiple outputs. They, too, demonstrate the augmenting the training data of image classifiers can improve generalization. 

\cite{teney2020learning} propose an approach using counterfactuals in training that does not rely on data augmentation: they argue that counterfactual pairs typically already exist in training datasets. Specifically, their approach relies on, firstly, identifying similar input samples with different annotations and, secondly, ensuring that the gradient of the classifier aligns with the vector between pairs of counterfactual inputs using the cosine distance as a loss function (referred to as \textit{gradient supervision}) (\textcolor{red}{this might be useful for our task as well}). 

In the natural language processing (NLP) domain, counterfactuals have similarly been used to improve models through data augmentation: \cite{wu2021polyjuice}, propose POLYJUICE, a general-purpose counterfactual generator for language models. They demonstrate empirically that augmenting training data through POLYJUICE counterfactuals improves robustness in a number of NLP tasks. 

\subsection{Link to Adversarial Training}

\cite{freiesleben2022intriguing} propose two definitional differences between Adversarial Examples (AE) and Counterfactual Explanations (CE): firstly, and more importantly according to the authors, the term AE implies missclassification, which is not the case for CE (\textcolor{red}{this might be a useful notion for use to distinguish between adversarials and explanations during training}); secondly, they argue that closeness plays a more critical role in the context of CE but confess that even counterfactuals that are not close might be relevant explanations. \cite{pawelczyk2022exploring} show that CE and AE are equivalent under certain conditions and derive upper bounds on the distances between them. 

\subsection{Closely Related}

\cite{guo2023counternet} are the first to propose end-to-end training pipeline that includes counterfactual explanations as part of the training prodeduce. In particular, they propose a specific network architecture that includes a predictor and CE generator network (\textcolor{red}{akin a GAN?}), where the parameters of the CE generator network are learnable. Counterfactuals are generated during each training iteration and fed back to the predictor network (\textcolor{red}{here we are aligned}). In contrast, we impose no restrictions on the neural network architecture at all. \textcolor{red}{NB: to ensure the one-hot encoding of categorical features is maintained, they simple use softmax (might be interesting for CE.jl)}. Interestingly, the authors find that their approach is sensitive to the choice of the loss function: only MSE seems to lead to good performance. They also demonstrate theoretically, that the objective function is difficult to optimize due to divergent gradients (\textcolor{red}{because partial gradients with respect to the classification loss component and the counterfactual validity component point in opposite directions}) and suffers from poor adversarial robustness. To mitigate these issues, the authors use block-wise gradient descent: they first update with respect to classification loss and then use a second update with respect to the other loss components (\textcolor{red}{this might be useful for our task as well}).

\cite{ross2021learning} propose a way to train models that are guaranteed to provide recourse for individuals with high probability. The approach builds on adversarial training (\textcolor{red}{here we are aligned}), where in this context adversarial examples are actively encouraged to exist, but only target attacks with respect to the positive class. The proposed method allows for imposing a set of actionable recourse ex-ante: for example, users can impose mutability constraints for features (\textcolor{red}{here we are aligned}). \textcolor{red}{NB: To solve their objective function more efficiently, they use a first-order Taylor approximation to approximate the recourse loss component (might be applicable in our case).}

\cite{luu2023counterfactual} introduce Counterfactual Adversarial Training (CAT) with intention of improving generalization and robustness of language models. Specifically, they propose to proceed as follows: firstly, identify training samples that are subject to high predictive uncertainty (entropy); secondly, generate counterfactual explanations for those samples; and, finally, finetune the model on the augmented dataset that includes the generated counterfactuals.

\bibliography{bib}
\bibliographystyle{icml2025}


%%%%%%%%%%%%%%%%%%%%%%%%%%%%%%%%%%%%%%%%%%%%%%%%%%%%%%%%%%%%%%%%%%%%%%%%%%%%%%%
%%%%%%%%%%%%%%%%%%%%%%%%%%%%%%%%%%%%%%%%%%%%%%%%%%%%%%%%%%%%%%%%%%%%%%%%%%%%%%%
% APPENDIX
%%%%%%%%%%%%%%%%%%%%%%%%%%%%%%%%%%%%%%%%%%%%%%%%%%%%%%%%%%%%%%%%%%%%%%%%%%%%%%%
%%%%%%%%%%%%%%%%%%%%%%%%%%%%%%%%%%%%%%%%%%%%%%%%%%%%%%%%%%%%%%%%%%%%%%%%%%%%%%%
\newpage
\appendix
\onecolumn
\section{You \emph{can} have an appendix here.}

You can have as much text here as you want. The main body must be at most $8$ pages long.
For the final version, one more page can be added.
If you want, you can use an appendix like this one.  

The $\mathtt{\backslash onecolumn}$ command above can be kept in place if you prefer a one-column appendix, or can be removed if you prefer a two-column appendix.  Apart from this possible change, the style (font size, spacing, margins, page numbering, etc.) should be kept the same as the main body.
%%%%%%%%%%%%%%%%%%%%%%%%%%%%%%%%%%%%%%%%%%%%%%%%%%%%%%%%%%%%%%%%%%%%%%%%%%%%%%%
%%%%%%%%%%%%%%%%%%%%%%%%%%%%%%%%%%%%%%%%%%%%%%%%%%%%%%%%%%%%%%%%%%%%%%%%%%%%%%%

\begin{tabular}{rrrrrrr}
    \hline
    \textbf{run} & \textbf{objective} & \textbf{lambda\_energy\_exper} & \textbf{lambda\_energy\_eval} & \textbf{generator\_type} & \textbf{mean} & \textbf{std} \\
    \texttt{Int64} & \texttt{String7} & \texttt{Float64} & \texttt{Float64} & \texttt{String7} & \texttt{Float64} & \texttt{Float64} \\\hline
    1 & full & 0.01 & 0.1 & ecco & \color[rgb]{0.151125, 0.307698, 0.0888533}{-1.18619} & 0.268524 \\
    1 & full & 0.01 & 0.1 & generic & \color[rgb]{0.151125, 0.307698, 0.0888533}{-1.27949} & 0.35677 \\
    1 & full & 0.01 & 0.1 & omni & \color[rgb]{0.151125, 0.307698, 0.0888533}{-1.3237} & 0.188125 \\
    1 & full & 0.01 & 0.1 & revise & \color[rgb]{0.7513246125949373, 0.3469720755837307, 0.2567079184820078}{-17.8176} & 4.62883 \\
    1 & full & 0.01 & 0.5 & ecco & \color[rgb]{0.1667141175359549, 0.3128850265182584, 0.09773898950577427}{-1.72938} & 0.606102 \\\hline
  \end{tabular}

\end{document}


% This document was modified from the file originally made available by
% Pat Langley and Andrea Danyluk for ICML-2K. This version was created
% by Iain Murray in 2018, and modified by Alexandre Bouchard in
% 2019 and 2021 and by Csaba Szepesvari, Gang Niu and Sivan Sabato in 2022.
% Modified again in 2023 and 2024 by Sivan Sabato and Jonathan Scarlett.
% Previous contributors include Dan Roy, Lise Getoor and Tobias
% Scheffer, which was slightly modified from the 2010 version by
% Thorsten Joachims & Johannes Fuernkranz, slightly modified from the
% 2009 version by Kiri Wagstaff and Sam Roweis's 2008 version, which is
% slightly modified from Prasad Tadepalli's 2007 version which is a
% lightly changed version of the previous year's version by Andrew
% Moore, which was in turn edited from those of Kristian Kersting and
% Codrina Lauth. Alex Smola contributed to the algorithmic style files.
